% Created 2019-01-19 Sat 15:52
% Intended LaTeX compiler: pdflatex
\documentclass[11pt]{article}
\usepackage[utf8]{inputenc}
\usepackage[T1]{fontenc}
\usepackage{graphicx}
\usepackage{grffile}
\usepackage{longtable}
\usepackage{wrapfig}
\usepackage{rotating}
\usepackage[normalem]{ulem}
\usepackage{amsmath}
\usepackage{textcomp}
\usepackage{amssymb}
\usepackage{capt-of}
\usepackage{hyperref}
\author{Philippe Carphin}
\date{\today}
\title{SSH3100 Sociologie de l'ingénieur}
\hypersetup{
 pdfauthor={Philippe Carphin},
 pdftitle={SSH3100 Sociologie de l'ingénieur},
 pdfkeywords={},
 pdfsubject={},
 pdfcreator={Emacs 26.1 (Org mode 9.1.13)}, 
 pdflang={English}}
\begin{document}

\maketitle
\tableofcontents


\section{SSH3100 Sociologie de l'ingénieur}
\label{sec:org6079eeb}
\subsection{Team notes for travail de session}
\label{sec:org08981be}

Notes for the team should go in this shared file

\url{file:///Users/pcarphin/Dropbox/PolyMtl\_AUT2018/SSH3100/SSH3100A\_TravailDeSession/Notes\_TravailSession.org}

There is also a google doc at \url{https://docs.google.com/document/d/1\_EFlC3fcEQfcdxR8pWZtQ0WZNIo54kQitghTT-efDUI/edit?usp=sharing}
\subsection{Team}
\label{sec:org8023013}
Oumaima Kidari <oumaima.kidari@polymtl.ca>
Jules Lefebvre <jltlefebvre@gmail.com>
Cristian Fredes Salvador <cristian.fredes-salvador@polymtl.ca>
Philippe Carphin <philippe.carphin@polymtl.ca>
Victor Walgenwitz <victor.walgenwitz@etu.unistra.fr>
\subsection{Évaluations}
\label{sec:orge7e90cb}
\subsubsection{Examen intra 20\%}
\label{sec:org9fd63b4}
\begin{center}
\begin{tabular}{ll}
2 questions & textes du recueil 1 à 5\\
6 questions & matière des cours 1 à 6\\
\end{tabular}
\end{center}
\subsubsection{Projet de session 40\%}
\label{sec:org772a0db}
\begin{itemize}
\item Doit être auprès d'un organisation ou entreprise
\item Doit concerner l'adoption d'une technologie
\end{itemize}

\begin{quote}
Les impacts d'une technologie dans une organisation \textbf{du point de vue des
utilisateurs}.

Ce qui importe est le point de vue des utilisateurs.
\end{quote}

\begin{enumerate}
\item Fiche de projet
\label{sec:org2c79082}
\begin{itemize}
\item Membres
\item 1 courriel pour rejoindre l'équipe
\item QUelle organisation
\item Qui on va interviewer
\end{itemize}
\end{enumerate}
\subsection{Cours}
\label{sec:org1aa8577}
\subsubsection{Cours 1}
\label{sec:org8d20138}
\begin{enumerate}
\item Présentation des étudiants
\item Présentation du Plan de cours et évaluations
\item Concepts du cours
\end{enumerate}
Voir section évaluations.
\subsubsection{Cours 2,3,4,5}
\label{sec:orgc274452}
Sur papier seulement :(
\subsubsection{Cours 6 \textit{<2018-10-15 Mon 12:46> } Impacts des nouvelles technologies dans la vie quotidienne}
\label{sec:orgb714a13}
\begin{enumerate}
\item Retour sur le cours 5:
\label{sec:org83e7f13}
\begin{itemize}
\item La technologie dans une organisation.
\item Finalité de Réponse \(\Rightarrow\) Production (P/S)
\item Production \(\Rightarrow\) Ressources \(\Rightarrow\) Notion de performance(Critère
d'éfficience, d'efficacité et d'économie)
\item Multiples représentations de l'\textbf{organisation}
\item Catégories d'impacts:
\begin{itemize}
\item Emploi : Postes, pas le travail effectué
\item Travail : Les tâches effectuées
\item Dynamique Organisationnelle
\end{itemize}
\end{itemize}
\item Mise en contexte Nouvelles technologies de l'information et de la communcation \textbf{(NTIC)}
\label{sec:org5e8ef17}
\begin{itemize}
\item Omniprésence des NTIC
\item Impacts au près des usagers + Appropriation (voir cours 3)
\item NTIC \(\Rightarrow\) génèrent différents effets:
-* Effet de complémentarité* (Ex: Cellulaire et téléphones intelligents vs
  Téléphones fixes.
\begin{itemize}
\item Genre les deux peuvent communiquer ensemble, c'est ça qu'il veut dire par complémentarité
\end{itemize}
\begin{itemize}
\item \textbf{Effet de substitution} : un disparaît au profit de l'autre (Ex: Agenda
remplacé par téléphones intelligents (lien à faire avec le cours sur la
convergence technologique).
\item \textbf{Effets de réduction} :  On utilise plus une technologie par rapport à
une autre. (Ex: Utilisation de ipod réduite par les téléphones intelligents)
\end{itemize}
\end{itemize}

\item Exemple d'impacts d'Internet
\label{sec:org5568123}
\begin{enumerate}
\item Vie familiale
\label{sec:orgc371f53}
\begin{itemize}
\item Diminution du temps passé en famille
\item Communication simplifiée
\item Facilité de maintenir le contact avec les membres éloignés
\item Contribue à une meilleure gestion du budget familial
On est mieu en mesure de trouver des escomptes
\begin{itemize}
\item Augmentation de l'épargne familiale
\end{itemize}
\item Augmentation du temps passé \ldots{} suivre des activités d'intérêt commun.
\end{itemize}
\item Vie Sociale
\label{sec:org5f4a99b}
Défini weirdement par le prof comme étant les interactions hors-famille
\begin{itemize}
\item Formation de nouveaux liens sociaux (Communautés).
Les communautés étaient plus formées par des facteurs comme la proximité et
autres facteurs.  Maintenant, ça devient plus des intérêts communs
\item Intérêts plus ou moins grands des dynamiques sociales (hors-frontières)
\item Intérêt pour les facettes d'autres cultures (Musique, Nourriture etc.)
\item Intérêt pour l'engagement dans diverses causes de nature sociale
\item Participation à des mobilisations citoyennes
\end{itemize}
\item Education
\label{sec:org0c999cf}
Nous allons distinguer les impacts qu'on peut caractériser comme \textbf{négatifs} et
\textbf{positifs}.

\begin{enumerate}
\item Positifs
\label{sec:org0343139}
\begin{itemize}
\item Facilite les recherches
\item Facilite une plus grande production scientifique et une plus grande diffusion scientifique.
\item Accès à un plus grand nombre de bibliothèques
\item Meilleure coordination du travail d'équipe
\item Offre plus importante de cours et de programmes
\item Autonomisation plus importante des participants
\end{itemize}

\item Négatifs
\label{sec:org3bbc678}
\begin{itemize}
\item Augmentation des formes de tricherie (plagiat, sous-traitance)
\item Augmentation des fausses informations (faux documents de référence, faux
articles)
\item L'éducation est devenue un objet de consommation qui répond à la réalité de
l'offre et de la demande. (PHIL: I don't know if this wasn't true before the
Internet. But he explained it in a nice way.  Education is supposed to have
some sort of vocation to teach what the teacher thinks he needs to teach vs
what the students want (as customers))
\end{itemize}

\item Enjeux relatifs de l'internet dans le milieu de l'Éducation
\label{sec:org16cf28e}
\begin{enumerate}
\item Confusion entre Donnée, Informations, Savoir
\label{sec:org26bf1ff}

\begin{itemize}
\item \textbf{Donnée} : fait brut observable (ex: 5\%)
\item \textbf{Information} : Une donnée contextualisée
Ex : Le taux de la banque du canada est égal à 1.5\%
\item \textbf{Connaissance} : Sachant que le taux de Bance du Canada mes 1.5\% et projette des
placements et d'investissements.  La connaissance requiert des
connaissances préalables.
\end{itemize}

\item Émergence d'un savoir Flou vs Savoir Précis
\label{sec:orge04ec29}
\item L'aspect marchand, ce qui fait qu'à l'absence de contrôle et vérification, il devient difficile d'en assurer la qualité
\label{sec:org05d30bb}
\end{enumerate}
\end{enumerate}
\end{enumerate}

\item Impacts de la \textbf{monétique}
\label{sec:org7cac56b}

Monétique : Contraction de "monnaie" et "informatique".

"La monnaie est réduite à \{0,1\}"

\begin{enumerate}
\item Impacts dans les institutions financières
\label{sec:orgd70869d}
\begin{itemize}
\item Amélioration du volume d'affaires
\item Concentration des cervices variés au sein d'une même entité
\item Diversité de l'offre des services
\item Augmentation de la concurrence
\item Augmentation des fraudes.
\end{itemize}

\item Impacts sur les clients
\label{sec:org29c2abb}
\begin{itemize}
\item Accès à nos comptes 24/7
\item Possibilité de générer des rendements
\item Augmentation des fraudes
\item Diversification de l'offre des produits financiers
\item Augmentation de l'endettement
\item Accès plus grand au crédit
\end{itemize}
\item Impacts sur les employés des institutions financières
\label{sec:org378e63b}
\begin{itemize}
\item Changement dans la nature des tâches
\item Changement dans les compétences requises
\item Introduction de nouvelles formes de rémunération (fixes et
variables(performance))
\end{itemize}
\end{enumerate}
\item Impacts de la \textbf{domotique}
\label{sec:orgf006657}
\begin{itemize}
\item Gestion optimisée des fonctions de gestion d'une maison (chauffage,
électricité)
\item Amélioration de la sécurité
\item Possibilité de diminuer les coûts de gestion
\end{itemize}
\begin{enumerate}
\item Domotique pré '90
\label{sec:org9bcad8b}
\begin{itemize}
\item Domotique = échec
\item Faible pénétration dans le domaine domiciliaire
\item Pas de lien entre les différents systèmes
\item Appropriation non réussie
\end{itemize}
\item Domotique post '90 : '90 jusqu'à aujourd'hui
\label{sec:orgadbad32}
\begin{itemize}
\item Introduction d'offre consolidées par le biais des systèmes de sécurité (ADT)
\item Cette stratégie va faciliter la pénétration et une plus grande appropriation.
\item Multiplicité d'offreurs.
\end{itemize}
\end{enumerate}
\end{enumerate}

\subsubsection{Cours 6 - Discussion de l'examen}
\label{sec:org6187d37}
2 questions sur les textes du recueil (textes 1 à 5) [6/20]
1 question par cours pour [14/20]
\begin{enumerate}
\item Cours 1
\label{sec:org8222f47}
Les 4 raisons qui justifient le cours
\begin{itemize}
\item Institutionnelles : BCAPG, OIQ
\item Sociétale : Omniprésence de la technologie
\item Professionnelles : Impacts
\end{itemize}

Notion de sociologie de la technologie.

Lien Technologie \textbf{leftrightarrow} Société
\item Cours 2 Illustration du lien T-S à travers trois périodes
\label{sec:orga96cf0f}
\begin{itemize}
\item Préindustrielle
\item Industrielle
\item Post industrielle
\end{itemize}

Différentes représentations de la technologie

Faits technologiquest : 2 faits
\begin{itemize}
\item Faits sociaux
\item Faits technologiques
\end{itemize}

Différentes caractéristiques des nouvelles technologies

\item Cours 3
\label{sec:org89f7f49}

Notion de convergence technologique

Modèle de dévelopement social des technologies
\begin{itemize}
\item Phase 1 Production sociale
\item Phase 2 Diffusion Sociale
\item Phase 3 Appropriation
\end{itemize}

\item Cours 4
\label{sec:org5678649}

Impacts des grands projets
\begin{itemize}
\item Notion d'impacts + Opérationalisation
\item Illustrer les impacts des grands projets selon des catégories
\item Modèle d'évaluation sociale des technologies.
\end{itemize}

\item Cours 5
\label{sec:org69edd21}
Voir retour sur le cours 56
\end{enumerate}

\subsubsection{Cours 7 - INTRA}
\label{sec:orga9159d0}
\subsubsection{Cours 8 - Résistance au changement et Mouvements sociaux}
\label{sec:org913120c}

\begin{enumerate}
\item Plan
\label{sec:org6f299e3}
\begin{itemize}
\item Partie 1 Résistance au changement
\begin{itemize}
\item Définition
\item Manifestations
\item Raisons
\item Résistance vue comme un comportement rationel
\end{itemize}
\item Partie 2 Mouvements sociaux
\begin{itemize}
\item Définition
\end{itemize}
\end{itemize}

\item Choses abordées avant le cours
\label{sec:orgc7cfccc}
\begin{enumerate}
\item Utilité des indicateurs :
\label{sec:orgac674b1}
S'assurer que les répondants interprètent ma question sur la même base.  Assurer
une consistance des réponses.
\item Exemple de grille d'analyse
\label{sec:org4c45363}

\begin{enumerate}
\item Sujet
\item Répondants à choisir, échantillon homogène
\item Revue de littérature
\item Énoncé de l'hypothèse
\item Catégories et sous-hypothèses
\item Indicateurs + Questionnaire
\item Administration des questions
\item Réponses = Collecte de données
\end{enumerate}

\begin{center}
\begin{tabular}{lllllll}
\hline
Catégorie &  & Pers A & Pers B & Pers C & Pers D & Total\\
\hline
Qualité & Amélioration du &  &  &  &  & \\
 & nombre de reprises &  &  &  &  & \\
 & ---------------------- &  &  &  &  & \\
 &  &  &  &  &  & \\
\hline
Contenu du & Élargissement &  &  &  &  & \\
Travail & des tâches &  &  &  &  & \\
 & ------------- &  &  &  &  & \\
 &  &  &  &  &  & \\
\hline
Communication & Augmentation des &  &  &  &  & \\
 & liens interdépartement &  &  &  &  & \\
 &  &  &  &  &  & \\
\hline
Productivité/ & Augmentatdion des &  &  &  &  & \\
rendement & livrables &  &  &  &  & \\
\hline
\end{tabular}
\end{center}

Ex: Supposons que notre H\(_{\text{0}}\) : Les adjointes administratives utilisant le
progiciel de gestion \(\Lambda\) connaissent une modification de leur performance.

\begin{itemize}
\item Adjointes administratives (répondants)
\item Progiciel de gestion \(\Lambda\) (technologie)
\item Modification de la performance (impacts)
\end{itemize}

On va transformer les réponses obtenues sur une échelle de Likeut (0-5, 0-10,
-5 à +5)

Analyse pour le rapport:
\begin{itemize}
\item Analyse interne
C'est l'analyse intra-entrevue et l'analyse inter-entrevue

\item Analyse externe
Comparer nos résultats à ceux de la littérature.
\end{itemize}

L'analyse intra-entrevue est associée à l'axe vertical du tableau.

L'analyse inter-entrevue est associée à l'axe horizontal du tableau.

C'est avec l'analyse inter-entrevue qu'on va confirmer ou infirmer l'hypothèse
générale.

Noter qu'on doit établir à partir de quel score les réponses indiquent un
impact significatif.

Pour pallier à la taille non-significative de notre échantillon, on va comparer
nos résultats aux résultats de la littérature.

\item Suite du cours
\label{sec:org2943924}

Cours 4,5,6:  Notion d'impacts, TCS (technologies dans le corps social?)

Dans les cours qui suivront, nous verrons :

Réactions générées par les TCS
\begin{itemize}
\item Oppositions
\item Résistance au chamement
\item Mouvements sociaux
\item Déviances
\end{itemize}

À la suite de ça, on va s'intéresser aux modes de gestion.  Comme ci-haut on a
une problématique, on va parler de modes de gestions
\begin{itemize}
\item Dispositifs de régulation
\item Gestion du changement
\item Normes institutionnelles (comme les normes ISO par exemple)
\end{itemize}

Les \textbf{modes de gestion} sont des modalités pour prendre en compte les réactions.

Les modes de gestion sucitent des contre-réactions
\begin{itemize}
\item Groupes de pression
\item lobby
\end{itemize}

On va terminer avec les enjeux connexes
\begin{itemize}
\item Lien technologie \(\leftrightarrow\) Production
\item Gestion des risques
\item Post-Taylorisme
\item Modèles
\begin{itemize}
\item Métier
\item Compétences
\end{itemize}
\item Transfert de technologie
\end{itemize}
\end{enumerate}

\item Le cours
\label{sec:org3df4f59}

\begin{enumerate}
\item Résistance au changement
\label{sec:orgc5aa1fb}
\begin{enumerate}
\item Définitions
\label{sec:orgc88ecdb}

\textbf{Résistance au changement} : Ensemble de réactions explicites ou implicites
visant à montrer son désacocrd par rapport à une technologie ou un projet.

\item Manifestations
\label{sec:org7a27d1b}

\begin{itemize}
\item Refus d'utiliser la technologie
\item Remise en question de la pertinence de la technologie
\item Accent mis sur les difficultés
\item Refus de participer à la mise en oeuvre
\item Recours à la convention collective

Retour sur cours 5 Income -> Outcome

Inc -> Input -> Processus -> output -> outcome -> Inc

Et l'implantation de technologies dans ce dessin = efficience (amélioration du
ratio input/output).
\item Remise en question des compétences de ceux (souvent c'est des ingénieurs) qui
implantent la technologie dans l'organisation.
\item Demander la réalisation d'études (pour retarder la mise en oeuvre).

\item Un autre enjeux qui est très important à considérer : \textbf{Remise en question de
la technologie} surtout si l'entreprise va bien.
\end{itemize}

\item Raisons sous-jacentes
\label{sec:org58055aa}

Pourquoi les gens résistent-ils au changement.  Qu'est-ce qui pousse les gens à
adopter cette posture.

\begin{itemize}
\item Peur de l'inconnu
\item Perte d'avantages
\item Besoin de sécurité
\item Refus de réinvestir dans un apprentissage
\item Perte de référenciel
\item Facilité dans l'habitude qui disparaît.
\end{itemize}

\item Résistance au changement = Acte rationel
\label{sec:org2a71110}

Illustration: Un diplômé universitaire 

Stade 1

=> Investissement (temps, argent)
   => Diplôme
      => Emploi, rémunération, poste
         => Statut social
=> Rendement

Pour avoir l'emploi, rémunération, poste et le statut social, il faut avoir
l'investissement.

Le changement touche le rendement.

Suite à ça, puisque le changemnt touche le rendement, il faut réinvestir pour
maintenir le rendement.

La résistance au changement doit être vue comme une protection de
l'investissement.

\textbf{L'idée} est que \textbf{la résistance au changement est une protection des
investissements et donc est un acte rationnel}.
\end{enumerate}


\item Mouvements sociaux
\label{sec:orga55972b}

\begin{enumerate}
\item Définitions
\label{sec:org6fab9b8}
\textbf{Mouvement social} :
\begin{itemize}
\item Action de groupe / Action collective
\item Portant sur / Défendant une cause
\item Cause d'intérêt général
\item But du groupe = faire triompher sa cause
\begin{itemize}
\item En modifiant l'état ou la nature du contexte social
\item En gardant le statut quo.
\end{itemize}
\end{itemize}

\item Visions du mouvement social
\label{sec:org6193fdb}

\begin{enumerate}
\item Vision de Sidney Tarrow
\label{sec:orgdde5f3e}
\begin{itemize}
\item Présence des élites
\item Les élites orientent les décisions collectives dans le sens qui leur est bénéfique.
\item Pour pouvoir se faire entendre, les citoyens ordinaires vont se mobiliser en
mouvement social pour s'opposer aux élites
\end{itemize}
\item Vision d'Alain Touraine
\label{sec:org744cfd5}
(Dans toute situation de pays, vous avez toujours au sein du mouvemnt social
dans son ensemble, des sous-groupes qui sont discriminés.)
\begin{itemize}
\item Présence de discrimination au sein des groupes sociaux
\item Des sous-groupes discriminés dans leur propre société (Mouvement de libération
de la femme, droits civiques au US)
\item Le sous-groupe discriminé va se mobiliser pour remettre en question l'ordre
établi.
\end{itemize}
\end{enumerate}
\item Contexte d'apparition
\label{sec:orgbea401f}
\begin{itemize}
\item Contestations sociales => initialement étaient portées par les sindicats +
sur les rapports de travail
\item Évolutions sociales contemporaines
\begin{itemize}
\item Diminution de l'influence des syndicats
\item Enjeux sociaux dépassaient le cadre du monde du travail + le cadre des frontières géographiques/territoriales
\end{itemize}
\end{itemize}

Tout ça fait que les mouvements sociaux sont devenus des cadres de mobilisation.
\item Caractéristiques des mouvements sociaux
\label{sec:orge098649}
\begin{itemize}
\item Le mouvement social est un \textbf{mouvement de masse} (qui n'est pas lié à une classe
sociale en particulier contrairement aux syndicats)
\item Le mouvement social \textbf{présente un projet alternatif} (une autre manière de
concevoir le monde, ou présente des solutions. On n'est pas uniquement dans un
mode de contestation pure)
\item Le mouvement social a tendance à se \textbf{mondialiser} (du fait des nouvelles technologies).
\end{itemize}
\item Évolution et signification du mouvement social
\label{sec:orga1b1dbd}
On distingue deux évolutions:
\begin{itemize}
\item Le mouvement social se transforme en parti politique
\begin{itemize}
\item Ex: les partis verts
\end{itemize}
\item Le mouvement social se transforme élargi sa base et devient (augmenter le
membership), assure sa légitimité, devient un acteur social important mais
non-politique.
\end{itemize}
\item Mouvement social et Nouvelles technologies
\label{sec:org4af98cb}

Effet paracoxal

\begin{itemize}
\item D'un côté
\begin{itemize}
\item Nouvelles technos -> Modification des sturectures sociales -> perte d'emploi
et destruction de secteurs d'activités -> perte de lien social
\end{itemize}

\item De l'autre côté
\begin{itemize}
\item No du fait des mouvement sociaux permettent de reconstruire du lien social
\item On s'engage dans des causess que l'on veut défendre.
\end{itemize}
\end{itemize}
\end{enumerate}
\end{enumerate}
\end{enumerate}

\subsubsection{Cours 9 Déviances autour des nouvelles technologies}
\label{sec:orgd8030e3}
\begin{enumerate}
\item Retour cours précédent
\label{sec:orgbd23f85}
Retour cours précédent
\begin{itemize}
\item Résistance au changment
\begin{itemize}
\item 
\end{itemize}
\end{itemize}

Vision de Tareau-something et la vision de Tourelle-something
We need to be able to contrast them

Caractéristiques des mouvements sociaux.
\begin{itemize}
\item Mondialisation
\end{itemize}
\ldots{}
\item Plan
\label{sec:org9b6d8bb}
\begin{enumerate}
\item XYZ
\begin{enumerate}
\item Définition de la déviance
\item Fondements sociologiques
\item 
\end{enumerate}
\end{enumerate}

\item Cours
\label{sec:org40f836c}

\begin{enumerate}
\item Partie 1
\label{sec:org5b0d79c}
\begin{enumerate}
\item Définition de déviance
\label{sec:org945cfd1}
\begin{itemize}
\item Effet indésirables ou inattendus du fait de l'appropriation de la technologie
\item Effets qui induisent des \ldots{} négatifs
\end{itemize}

\begin{enumerate}
\item Déviance
\label{sec:orge7905f1}

\begin{enumerate}
\item On pose des actions qui ont des répercussions négatives sur soi-même.
\label{sec:orgaa3f201}
Poster des vidéos de conneries qu'on a fait quand on était drunk.
\item On pose des actions qui ont des répercussions sur les autres
\label{sec:org8ac5c9a}
Scammers indiens, voler des renseignements confidentiels.
\end{enumerate}
\end{enumerate}

\item Fondements sociologiques de la déviance
\label{sec:org33c5ce0}

Qu'est-ce qui explique l'écart par rapport à la norme sociale
\begin{itemize}
\item Sociologie \(\Rightarrow\) Norme Norme sociale impose une régulation \(\rightarrow\) Contrôle social
\item Absence de contrôle sociale favorise la déviance
\item Absence du regard des autres ((sur internet)) \(\uparrow\)
\end{itemize}
((Note sur le fait que les distances ne font plus de différences)).

\item Évolution de la déviance
\label{sec:orgcd9b93d}

\begin{enumerate}
\item Lien Appropriation \(\Leftrightarrow\) Déviance
\label{sec:org920d6b4}
Appropriation:
\begin{itemize}
\item Structure d'accueil
\item Usagers
\item Contextualisée
\end{itemize}

Technologie \(\Rightarrow\) Appropriation des impacts (positifs et négatifs)

Rapport T \(\leftrightarrow\) S \(\Rightarrow\) Évolution du lien T \(\leftrightarrow\) S 

Tout ça \(\Rightarrow\) Avec le temps de nouvelles formes de déviances.

\item Illustration de l'évolution de la déviance par la délinquance informatique
\label{sec:orgcb42d48}

3 périodes
\begin{enumerate}
\item Naissance de la microinformatique
\begin{itemize}
\item Actions/Comportements de déviances = vol de programmes
\end{itemize}
\item Naissance de d'une architecture locale (réseaux locaux)
\begin{itemize}
\item VOl de données dans le serveur local
\end{itemize}
\item Naissance et développement du Web
\begin{itemize}
\item Actions + Comportements = Virus, Hacking
\end{itemize}
\end{enumerate}
\end{enumerate}

\item Postulats relatifs à la déviance
\label{sec:orgae44af2}

\begin{enumerate}
\item Toute entité (individu, Organisation, État) possède des informations qui
intéressent les autres
\item Ces informations sont stockées et \ldots{} les systèmes informationels \ldots{}
\item Aucun système informatique n'set sure à 100\%.  Le risque nul n'existe pas.
Tout système comporte des failles
\item Toute entité ayant un intérêt et des connaissances peut donc s'introduire
dans le système et commettre ainsi un acte de déviance.
\item Lentité le fera d'autant plus facilement que le risque de se faire prendre
est mince.
\end{enumerate}

\item Ambivalence de la société face à la déviance
\label{sec:orgb279456}

Acteurs sociaux \(\Rightarrow\) Individus, Organisations, États

\(\Downarrow\)

\(\Rightarrow\) Posent des actes de déviance
\begin{itemize}
\item Incapacité à en faire la promoiton dans la mesure où c'est répréhensible
\end{itemize}

\(\Rightarrow\) Subissent les conséquences
\begin{itemize}
\item Difficultés à se plaindre ((Si on se plain on se condamne soi même parce que
nous aussi commettons des actes de déviances))
\end{itemize}
\end{enumerate}

\item Partie 2 Incursion dans la vie privée
\label{sec:org9bc97f0}
\begin{enumerate}
\item 2.1 Dispositifs légaux
\label{sec:org179826c}
\begin{itemize}
\item Différence entre les divers pays
\item Exemple Canada/Québec
\begin{itemize}
\item Loi sur la protection des informations personnelles.j
\end{itemize}
\end{itemize}


\item 2.2 Évolution technologique et acteurs sociaux
\label{sec:org0ac1a7a}
\begin{itemize}
\item Illustration dans le domaine de la télévision
\begin{enumerate}
\item Modes de production et de diffusion de la télévision sont limités

Limites technologiques

\begin{itemize}
\item Nombre de canaux limités

\item T.V appartient à l'état

\item Les émissions sont diffusées à des périodes données

\item ((La télé a un rôle social important)
\end{itemize}

\item Évolution technologique (fibre optique, etc)
\begin{itemize}
\item Les modes de produciton et de diffusion de la télévision vont changer.
\item Les capacités technologiques induisent \ldots{}
\begin{itemize}
\item Quantité exponentielle de chaînes

\item TV appartient au secteur privé
\item Le privé tient à rentabiliser la chaîne
\item \(\Rightarrow\) Modification des contenus (Mise en relief du citoyen \(\Lambda\))
\begin{itemize}
\item Émergence de la télé-réalité.
\item \(\Rightarrow\) On assiste au \textbf{décloisonnement entre la vie privée et la vie
publique}.
\end{itemize}
\end{itemize}
\end{itemize}
\end{enumerate}
\end{itemize}
\end{enumerate}
\end{enumerate}
\end{enumerate}
\subsubsection{Cours 10 Modes de gestion des oppositions et des déviances}
\label{sec:org1088bb8}
\begin{enumerate}
\item Plan
\label{sec:org77637f9}
Partie 1 Différents modes
\begin{enumerate}
\item Dispositifs de régulation
\begin{itemize}
\item Definition
\item Composantes
\item Limites
\end{itemize}
\item Normes ISO
\begin{itemize}
\item Definition
\item Contexte
\end{itemize}
\end{enumerate}
\item Rappel Cours 9
\label{sec:orgd12c7d8}
Notion de déviance \(\Rightarrow\) Line avec Fondements Sociologiques
\(\rightarrow\) Appropriation

Caractère évolutif 
(( Appropriation + Caractere evolutif \(\rightarrow\) Lien à faire entre les cours 3 et 9))
\begin{itemize}
\item On a pris comme exemples de la délinquance informatique

\item Postulats et Raisons

((On ne peut pas éviter la déviance))

\item Ambivalence des entités sociales face à ce phénomène.

\item La déviance se matérialise sous deux formes
\begin{itemize}
\item Subir la conséquence des actes \textbf{d'une autre personne}
\item Subir la conséquence des actes \textbf{de ses propres actes}
\end{itemize}

\item Incursion dans la vie privée
\end{itemize}

\textbf{Ceci amène à considérer les modes de gestion.}
\item Cours
\label{sec:orgce667a2}
\begin{enumerate}
\item Partie 1 Différents modes
\label{sec:orgd37b6f5}
\begin{enumerate}
\item Dispositifs de régulation
\label{sec:orge909a59}
\begin{enumerate}
\item Définition
\label{sec:org9363062}

Ensemble de règles formelles ou informelles visant à encadrer les comportements
des acteurs sociaux.

\textbf{Exemples}
\begin{itemize}
\item Lois
\item Règles culturelles
\end{itemize}
((Ces dispositifs de régulation devienntent des normes, des repères))
\item Composantes
\label{sec:orgfc67596}
En sociologie, on distingue trois types de régulation.
\begin{enumerate}
\item Régulation de contrôle
\label{sec:org65a1be3}
Une entité qui établit des règles qui s'appliquent à tous.

Cette entité \textbf{a un pouvoir de sanction}.

\textbf{Exemples}
\begin{itemize}
\item L'état
\end{itemize}

\item Régulation Mixte
\label{sec:orgbbe771c}
L'entité de contrôle délègue son pouvoir à une organisation, généralement
paritaire qui établit les règles pour un secteur.

\textbf{Exemples}
\begin{itemize}
\item Au québec
\begin{itemize}
\item CSST
\item CNT
\end{itemize}
\item Au canada
\begin{itemize}
\item CRTC
\end{itemize}
\end{itemize}

\{\{J'attendais qu'il commence à parler de l'OIQ et pourquoi il parlait du CRTC
mais dans le fond, l'OIQ régule les ingénieurs, pas vraiment une technologie
alors que la CSST régule des technologies dans les environnements de travail et
le CRTC qui régule les technologies de communciation.\}\}
\item Régulation autonome
\label{sec:org751264f}

Un groupe qui se dote de ses propres règles, celles-ci s'appliquent uniquement
aux membres du groupe.

\textbf{Exemples}
\begin{itemize}
\item Équipe de travail dans un cours
\end{itemize}
\end{enumerate}
\item Application et limites des 3 dispositifs de régulation
\label{sec:orga1babc4}
\begin{enumerate}
\item Régulation de contrôle : ((Fait un affaire socratique pour nous faire dire
que "NON" la régulation autonome ne peux pas opérer sur l'internet.

\item Régulation autonome : ((Socratique pour dire que "Non" parce que si l'autre
est pas dans mon groupe internet whatever,

\item Régulation mixte : ((Même la régulation mixte ne peut pas donner de réels
résultats.
\end{enumerate}

((On voit que les systèmes de régulations ne peuvent pas tout faire))
\item Plusieurs limites
\label{sec:orgcf96489}
\begin{itemize}
\item La \textbf{technologie évolue plus vite que les lois}
\item Le web ne s'arrête pas aux frontières alors que les lois ne s'appliquent
uniquenment au sein d'un espace géographique.
\item Les organisations policières ne mettent pas les ressources adéquates.
\end{itemize}
\end{enumerate}



\item Normes ISO
\label{sec:orge7c11d2}

\begin{enumerate}
\item Définition
\label{sec:orgebe13c1}
Ensemble de règles basées sur des dimensions organisationnelles.  Pour être plus
précis: de gestion organisationnelle
\begin{itemize}
\item Qualité
\item Environnement
\end{itemize}
visant à établir les pratiques de référence (les "meilleures pratiques").

\textbf{NOTE} La norme ISO porte le même nom que l'organisation ISO.

\textbf{NOTE} Les normes ISO sont volontaires.  Une organisation décide de s'y
conformer.  \{\{Ça peut être préférable et obligatoire en pratique.\}\}

\item Contexte
\label{sec:org507574f}

\begin{itemize}
\item Étant donné que c'est volontaire, pourquoi les organisation adoptent-elles les
normes ISO?

\item Le constat d'une période industrielle à poste industrielle ((cours 2))
\begin{itemize}
\item Reconnaissance de la mondialisation \(\rightarrow\) Règles de l'OMC
\item Répartition des zones de production et de consommation
\item Concurrence mondialisée alors que les contextes sont différents.
\end{itemize}

\(\Rightarrow\) \textbf{Nécessité d'évaluer les organisations sur les mêmes bases}

\item L'économie de la qualité fait que les compagnies veulent la certification ISO
qui garantit la qualité de leur produit.
\end{itemize}
\item Démarche d'élaboration par iso
\label{sec:org8ff800d}

\begin{itemize}
\item Mise en place d'un comité
\item Comité (Experts industrie, Société Civile, États).
\item ISO Énonce/rédige un mandat
\item comité reçoit le mandat établit un plan de travail
\item Le comité élabore la proposition, qui est discutée et testée
\item \textbf{Le comité établit le cahier de charge}
\item Quand le cahier de charge est établi, il revient à l'organisation ISO d'en
assurer la mise en oeuvre.
\end{itemize}

\item Démarche de certification
\label{sec:orgbec1151}

Une entreprise/organisatin voulant être certifiée (ex: ISO 9001 (qui porte sur
la qualité)).

L'organisation fait une requête auprès de ISO ou d'un auditeur agréé.


\begin{enumerate}
\item Audit de l'organisation ((sur la base du cahier de charge)).
\item Action à mettre en oeuvre pour se conformer.
\item \textbf{Plan de mise en oeuvre} ((bonne vieille gestion de projet))
\begin{itemize}
\item avec des jalons (milestones)
\end{itemize}
\item Exécution (ex implantation des pratiques de qualité)
\item Suivi conformité et certification
\end{enumerate}

Tout ce processus est à la charge de l'organisation/entreprise requérente

\item Avantages d'être certifié ISO
\label{sec:org4a3597b}

\begin{itemize}
\item On se démarque des par rapport aux entreprises concurrentes.
\item C'est souvent un prérequies pour certaines soumissions
((Ex: pour la stm, les "rammes" de métro \{\{pas sur quel mot il a dit\}\},
l'appel d'offre demandait une compagnie ISO bla bla))
\item C'est une amélioration substantielle des pratiques de gestion \(\Rightarrow\) Efficience.
\end{itemize}
\end{enumerate}

\item Gestion du changemnt
\label{sec:org4545433}
\begin{enumerate}
\item Définition
\label{sec:org7107c03}
\begin{itemize}
\item Gestion fait référence aux activités de P.O.D.C. (Planifier, Organiser,
Diriger, Contrôler)

\item Gestion du changement correspond aux activités de PODC permettant à une
organisation de se transformer, de s'adapter. De passer d'une situation
non-performante à une situation perforamnte.
\end{itemize}

\noindent\rule{\textwidth}{0.5pt}

\item Gestion du changement et Gestion de projet
\label{sec:org196949e}

\begin{itemize}
\item Gestion du changement = gestion de projet
\begin{itemize}
\item Contrainte de temps : Échéancier
\item Obligation de résultats
\item Des ressources dédiées
\item Perspective multidisciplinaire
\end{itemize}
\item Nécessité de distinguer Projet vs Gestion Conventionnelle
\item Comment se déploie un projet
\begin{itemize}
\item Idéation \{\{WTF is that word lol\}\}
\item Conception/Étude
\item Planification/Réalisation
\item Tests
\item Transfert a l'exploitant
\end{itemize}
\end{itemize}

\textbf{Le PMI} project management institute, PM BOOK

\item Activités
\label{sec:org95c79d0}
\begin{itemize}
\item Enquête auprès de ceux qui doivent changer leurs pratiques.
\item Analyse des besoins des employés
\item Établir un plan de formation
\item Établir un plan de communication
\item Mettre en place un comité paritaire
\item Mettre un place un cservice d'accompagnement
\item Mettre en place un système d'incitatifs
\end{itemize}
\item Cadres théoriques
\label{sec:org46ec88f}

Comment les acteurs sociaux pensent et structurent les projets de changement.
\begin{enumerate}
\item Tableau
\label{sec:orgce040e4}
\begin{center}
\begin{tabular}{llll}
\hline
Cadres théoriques & Technologie & Acteur & Organisation\\
\hline
\hline
Déterminisme & Objet Abstrait & Rôle faible & Déterminée\\
 &  & subit & \\
\hline
Théorie du & Offre des & Exerce un choix & Variable\\
choix rationnel & Possibilités & d'adaptation & Modératirice\\
 & d'adaptation &  & \\
\hline
Internationalisme & Combinaison de & En interaction & Réseau\\
 & Facteur humain & base du succès & d'acteurs\\
 & + &  & \\
 & Facteurs matériels &  & \\
\hline
\end{tabular}
\end{center}

\begin{enumerate}
\item Déterminisme = Le succès d'une technologie est déterminé uniquement par les
fonctions de la technologie.

((on met l'accetn sur l'outil, si j'ai le bon outil tout va vien aller))

\item Théorie du choix rationnel

\begin{itemize}
\item Une technologie offre des possibilités.
\item Celles-ci vont être utilisées par les acteurs sociaux.
\end{itemize}

((être rationnel signifie prendre la décision qui maximise bla bla, ce qui
peut mener à une situation ou tout le monde tire la couverture de son bord))

\item Interactionnisme

\begin{itemize}
\item La technologie est appropriée
\item Réseau d'acteurs
\end{itemize}

\(\Rightarrow\) Concensus

\begin{enumerate}
\item Interaction entre les acteurs sociaux
\item Négociation
\item Concensus
\end{enumerate}
\end{enumerate}

((Les gestionnaires et ingénieurs peuvent faire l'erreur de tomber dans le
déterminisme ou la théorie du choix rationnel et négliger l'appropriation de la technologie.
\end{enumerate}
\end{enumerate}
\end{enumerate}


\item Partie 2 Groupes de pression
\label{sec:orgdec96fc}

\begin{enumerate}
\item Groupes de pression
\label{sec:orga2e0f34}

Cours 4,5,6 \(\Rightarrow\) Impacts, Techno, TCS

\(\Downarrow\)

Cours 8,9 \(\Rightarrow\) Réactions, Oppositions/Déviances

\(\Downarrow\)

Cours 10 \(\Rightarrow\)  Modes de gestion, de régulation Règles \(\Rightarrow\) Suscitent des
contre-réactions, ex: Groupes de pression et Lobbys.

\begin{enumerate}
\item Définition de groupes de pression (Lobby)
\label{sec:orgfff7f53}

Un groupe de pression représente un groupe d'acteurs sociaux représentant des
intérêts privés.

\textbf{But}: Agir auprès des déscideurs afin que les dispositifs (règles)
touchent/affectant le moins possible les intérêts qu'ils défendent.

\item Ressources et acteurs
\label{sec:org70be5e5}

\begin{itemize}
\item GP bénificient des ressources de secteurs défendus
\item Plus un secteur est économiquement important, plus le G.P. dispose de
ressources.
\item Financer les activités sociales (tournois, concours, fête, etc (ex: Le
GrandPrix de Montréal était financé jusqu'à y'a pas longtemps par les
compagnies de cigarettes)).
\item Acheter des encarts publicitaires
\item Financer des études, des experts
\end{itemize}

\item Deux visions des groupes de pression
\label{sec:org29795cc}

Les deux visions voient le groupe de pression comme un mécanisme pour faire
valoir des intérêts spécifiques.
\begin{enumerate}
\item Vision corporatiste
\label{sec:org175df10}

Le groupe de pression a légalement le droit d'exister ((Il a parlé de M. Landry
qui a fait passer une loi pour donner un cadre législatif aux groupes de
pression)).

Analyse d'Adam Smith, il a écrit un livre important en économie.
\begin{itemize}
\item Chaque nation a une dotation natureelle en ressoursces
\item Étant donné ces dotations, je vais exploiter mes ressoursces et les échanger
avec d'autres nations.  Exploiter les ressources à moindre coûte aussi
\end{itemize}

\item Vision démocratique
\label{sec:org7512f61}

La vision démocratique nous sommes dans un système de démocratie: 1 personne -
une voix.

Élection des élus qui  rennent des décisions auprès de tous.

Les intérêts spécifiques sont pris en compte dans le processus démocratique
((donc on n'a pas besoin des GP pour veiller à des intérêts spécifiques)).

\item Tableau
\label{sec:org4990c37}

\begin{center}
\begin{tabular}{ll}
Vision Corporatiste & Vision démocratique\\
\hline
GP est légal et légitime & GP est certes légal, mais illégitime\\
Basée sur l'analyse d'Adam Smith & \\
En laissant les individus s'occuper des intérets particuliers, on sert l'intérêt généralement & \\
\(\Rightarrow\) Les GP s<occupent des intérêts sectoriels et donc servent l'intérêt général & \\
\end{tabular}
\end{center}



Les deux ont en partie raison et en partie tort.

\begin{itemize}
\item Les intérêts particuliers ne sont pas toujours servis par la démocratie à
cause de la bullshit de ligne de partie.  Le député de ton cartier ne vote pas
selon les besoins mais bien selon la ligne de parti.
\end{itemize}

d'autre part

\begin{itemize}
\item La vision corporatiste est empreinte de malversation ((commission charbonneau)).
\end{itemize}


Les deux visions nous donne une image de groupes de pression.
\end{enumerate}

\item Arguments pour et contre
\label{sec:orgcd510ba}

\begin{enumerate}
\item Contre
\label{sec:org941fc2a}
\begin{itemize}
\item Le groupe de pression participe à différentes pratiques délictuelles
\item Les gp font peu état de transparence
\item Les GP établissent des rapports de force vis à vis les autres acteurs sociaux
(ex: chantage à l'emploi).
\end{itemize}

\item Pour
\label{sec:org4d12bd8}
\begin{itemize}
\item Peuvent contribuer au développement d'un secteur d'activité. Ex: Le secteur du
multimédia au Québec (ubisoft, nanoquébec).
\item Contribuent aussi développement des connaissances en finançant les recherches.
\item Peuvent enrichir le débat social
\end{itemize}
\end{enumerate}
\end{enumerate}
\end{enumerate}
\end{enumerate}
\end{enumerate}
\subsubsection{Cours 11 Notes de Saïf et de Cristain}
\label{sec:orgb77347d}
\subsubsection{Cours 12, 13 manquants}
\label{sec:org91c0053}

\subsection{Retour sur l'intra}
\label{sec:org5aee3c2}

\subsubsection{Question 1}
\label{sec:orgc04cde4}
Délégations : Citoyens vs État
2 Délégation : Profane vs profs + chercheurs + scientifiques

Les cultures s'approprient
\subsubsection{Question 2}
\label{sec:orgaa3e412}

Première page de l'article.  L'auteur indique trois phases

Constitution d'un marché (concepteurs et R\&D qui agissent)

Diffusion (Expérimentation et tests). C'est la période des retours et
coinventions

Impacts.

\subsubsection{Question 1}
\label{sec:org1034a90}

Modèle de développement social de la technologie

3 représentations de l'appropriation
\begin{itemize}
\item Utiliser la technologie conformément à sa conception
\item Adapter la technologie à ses propres besoins
\item La technologie devient un prétexte pour de nouvelles interactions sociales
\end{itemize}

\subsubsection{Question 2}
\label{sec:org33c7449}

\begin{itemize}
\item Impacts sur l'emploi (reference aux postes) \(\checkmark\)
\item Impacts sur le travail (ref aux taches) \(\checkmark\)
\item Impacts sur la culture (normes et pratiques sociales)
\end{itemize}

\subsubsection{Question 3}
\label{sec:org4fefa4e}

\begin{itemize}
\item Performance sociale \ldots{}
\end{itemize}

\subsubsection{Question 4}
\label{sec:org6d720af}

3/3 !!!

\subsubsection{Question 5}
\label{sec:orgaaefcab}

Objectifs
\begin{itemize}
\item Parler des interactions générées dans le corps social (la techno modifie le
corps social \ldots{})
\end{itemize}
Relation
\begin{itemize}
\item La société influence la techno et vice versa
\end{itemize}

Exemple
\begin{itemize}
\item 
\end{itemize}
\subsection{Textes du recueil}
\label{sec:org28a2128}
\subsubsection{Lejeune, M. L'apport de la sociologie de la technologie à la professionnalisation de l'ingénieur.}
\label{sec:orgae5f929}

La professionalisation de l'ingénieur, sous l'angle de la sociologie de la
technologie, s'iscrit dans les énoncés de principe et les normes canadiennes
d'agrément des programmes universitaires d'ingénierie. La question du
féveloppement des technologies (production, diffusion et appropriation) se
rattache en tous ses points aux qualistés personnelles et sociales de
l'ingénieur auxquelles féfèrent les écoles d'ingénierie et leurs organismes de
régulation. La profession d'ingénieur s'enracine d'ailleurs dans les milieux
industriels et sociaux ede plus en plus sensibles aux dimesnions sociales de la
technologie, considérant les nouvelles dynamiques qui s'y rattachent en regard
des impacts de la technologie sur la société et dans les entreprises.

\subsubsection{Texte 7 Les \textbf{Trois leviers stratégiques} de la réussite du changement technologique}
\label{sec:org09f9966}
\end{document}
