% Created 2019-01-19 Sat 15:43
% Intended LaTeX compiler: pdflatex
\documentclass[11pt]{article}
\usepackage[utf8]{inputenc}
\usepackage[T1]{fontenc}
\usepackage{graphicx}
\usepackage{grffile}
\usepackage{longtable}
\usepackage{wrapfig}
\usepackage{rotating}
\usepackage[normalem]{ulem}
\usepackage{amsmath}
\usepackage{textcomp}
\usepackage{amssymb}
\usepackage{capt-of}
\usepackage{hyperref}
\author{Philippe Carphin}
\date{\today}
\title{Économie de l'ingénieur}
\hypersetup{
 pdfauthor={Philippe Carphin},
 pdftitle={Économie de l'ingénieur},
 pdfkeywords={},
 pdfsubject={},
 pdfcreator={Emacs 26.1 (Org mode 9.1.13)}, 
 pdflang={English}}
\begin{document}

\maketitle
\tableofcontents


\section{Économie de l'ingénieur}
\label{sec:org2910b38}

\subsection{Cours}
\label{sec:org1e90c39}

\subsubsection{Cours 1}
\label{sec:orgfeb69f9}
\textbf{J'ai manqué ce cours}.
Le sujet est les états financiers.
\begin{enumerate}
\item Plan
\label{sec:org73eb49f}
\begin{itemize}
\item Introduction
\item Objectifs Généraux du cours
\item Formes d'entreprise
\item États financiers (présentation)
\begin{itemize}
\item État de la situation financière
\item État des résultats
\item État des variations des capitaux propres
\item État des flux de trésorie (prochain cours)
\end{itemize}
\item Exemple
\end{itemize}

\item Formes d'entreprise
\label{sec:org8defaa8}
\begin{enumerate}
\item Formes Juridiques
\label{sec:orgebb207c}
\begin{enumerate}
\item Entreprise individuelle
\label{sec:org7c85729}
\item Société de personnes
\label{sec:org1d0d76f}
\item Société de capitaux
\label{sec:org4bdeb66}
\item Coopérative
\label{sec:org946ac37}
\end{enumerate}
\item Formes économiques
\label{sec:org3e1817e}
\begin{enumerate}
\item Entreprise industrielle
\label{sec:org1906190}
\item Entreprise comemrciale
\label{sec:orgb7e669f}
\item Entreprise de services
\label{sec:org4138b29}
\end{enumerate}

\item Tableau sur les formes juridiques
\label{sec:org2f56567}
\end{enumerate}

\item Comptabilité
\label{sec:org77ab94d}
\begin{enumerate}
\item Définition
\label{sec:orgda8006c}
C’est unsystème d’informationfinancière permettant de:
-enregistrerles opérations d’une entreprise
-lesregrouper
-les classer
-les présenter(dans les états financiers É/F) pour eninterpréter les résultats.
\item Rôle
\label{sec:org354b2db}
Transformerl’ensemble des données financières en informations utiles et
pertinentes(É/F) pour les divers groupes qui forment l’environnement économique
de l’entreprise.

Depuis janvier 2011, les sociétés ouvertes (cotées en bourse) doivent adopter
les Normes internationales d'information financière (IFRS).
\end{enumerate}

\item Notions de base
\label{sec:orgb396265}
\begin{itemize}
\item Personnalité de l’entreprise (entité séparée)
\item Indépendance des exercices et rattachement des
charges aux produits
\item Coût historique (valeur d’acquisition)
\item Comptabilisation (constatation des produits et charges)
\item Permanence des méthodes comptables
\item Prudence
\end{itemize}

\item étapes du système comptable
\label{sec:org061f774}
\textbf{NOTE} Seule la partie «Établissement des états financiers» sera traitée dans ce cours
\begin{enumerate}
\item enregistrement des opérations
\label{sec:org2570c40}
journal général

\item report dans le grand livre général
\label{sec:org7234684}
comptes en T

\item Établissement de la balance de vérification
\label{sec:orge017458}

\item Régularisation et clôture des comptes
\label{sec:org60d90d1}
Charges, Produits

\item Établissement des états financiers
\label{sec:org2217ebf}
\begin{itemize}
\item état des résultats
\item état des variations des capitatux propres
\item état de la situation financière
\item état des flux de trésorerie
\end{itemize}
\end{enumerate}

\item Compte en T
\label{sec:orgb98319b}
\item Présentation des états financiers
\label{sec:orgefc13ef}

\begin{enumerate}
\item État des résultats
\label{sec:orgb14f678}
\begin{enumerate}
\item Tableau
\label{sec:org9a252cc}
\begin{center}
\begin{tabular}{ll}
<Nom compagnie> & \\
État des résultats & \\
du j j-mm-aaaa au jj-mm-aaaa & \\
\hline
\uline{produits} & \\
+ Ventes & x\\
- Couts des ventes & (x)\\
\textbf{Résultats bruts} & XX\\
+ Produits de location & x\\
+ Produits d'intérêts & x\\
+ Gains sur disposition (prix de vente - valeur comptable) & x\\
\uline{Charges} & \\
+ Écliectricité & x\\
+ Salaires & x\\
+ Assurances & x\\
+ Amortissement (Sans effet sur la trésorie) & x\\
+ Dépréciation des comptes client (Sans effet sur la trésorie) & x\\
+ Pertes sur disposition d'actif & x\\
\uline{Resultat net} & \\
= Sum(Produits) - Somme(Charges) & XX\\
\uline{Resultat net après impôts} & \\
= Resultat net - impot & XX\\
\end{tabular}
\end{center}
\item Composantes
\label{sec:orgede1e95}
\begin{itemize}
\item Actifs(A):ressources économiques que l’entreprise possède ou sur lesquelles
elle exerce un contrôle et qui devraient lui procurer des avantages
économiques maintenant ou dans le futur.

\item Passifs(P):obligations qu’a l’entreprise envers des tiers et dont le
règlement se fera par transfert d’actifs, la prestation de services ou tout
autre avantage.

\item Capitauxpropres(C): mise(s) defonds du(des) propriétaire(s) auxquelles
s’ajoutent les résultats non distribués aux actionnaires sous forme de
dividendes (société par actions) et le surplus d’apport.
\end{itemize}
\item Vérifications
\label{sec:orgb4b94bb}
Les éléments ici influencent l'état de variations des capitaux propres et la
situation financière qui est vérifié avec l'équation comptable.
\end{enumerate}
\item État des variations des capitaux propres
\label{sec:org1ffe1a0}
\begin{enumerate}
\item Tableau
\label{sec:org8d353e2}
\begin{center}
\begin{tabular}{llll}
<nom de la compagine> &  &  & \\
État des variations des capitaux propres &  &  & \\
Période terminée le <jj-mm-aa> &  &  & \\
\hline
 & Capital Social & RND & Surplus d'apport\\
\hline
Solde de début & a & b & c\\
\hline
Surplus d'apport de la période &  &  & h\\
Emission d'actions & d &  & \\
Rachat d'actions & (e) &  & \\
Resultats de la periode &  & f & \\
Dividendes déclarés &  & (g) & \\
\hline
Solde de fin & a + d - e & b + f - g & c + h\\
\hline
\end{tabular}
\end{center}

\item Explications
\label{sec:orga957554}
On prend a, b, c, de l'état des résultats.

Émettre des actions augmente le capital social.  Les RDN de fin est notre RND
précédent plus nos résultats de la période moins la partie des résultats qu'on a
versé en dividendes.  Finalement le surplus d'apport est le surplus d'apport n-1
plus le surplus d'apport que les propriétaires on mis durant la période.
\item Vérifications
\label{sec:org6f7f3a0}
Les éléments ici influencent l'état de la situation financière qui est vérifié
avec l'équation comptable.
\end{enumerate}
\item État de la situation financière
\label{sec:org07f91fe}
\begin{enumerate}
\item Tableau
\label{sec:orgb233aee}
\begin{center}
\begin{tabular}{lll}
\hline
<Nom de la compagnie> &  & \\
État de la situation financière &  & \\
Au 31 décembre 2017 &  & \\
\hline
 & 2017 & 2018\\
\hline
\uline{Actifs} &  & \\
\hline
\uline{Actifs courants} (< 12 mois) &  & \\
Encaisse & x & x\\
Stocks & x & x\\
Comptes Clients & x & x\\
Placements à court terme & x & x\\
(Provision pour dépréciation des CC) & (x) & (x)\\
Produits à recevoir & x & x\\
Charges payées d'avance & x & x\\
\textbf{Total actifs courants} & XX & XX\\
\hline
\uline{Actifs non-courants} &  & \\
Immobilisation corporelles & x & x\\
Équipement/Machines & x & x\\
(Amortissement cumulé Équipement/Machine) & (x) & (x)\\
Immeubles & x & x\\
(Amortissement cumulé immeubles) & (x) & (x)\\
Immobilisations incorporelles (breuvets, licences) & x & x\\
Placements à long terme & x & x\\
\textbf{Total actifs non-courants} & XX & XX\\
\hline
\textbf{Total Actifs} & XXX & XXX\\
\hline
Passifs et Capitaux propres &  & \\
\hline
\uline{Passifs courants} &  & \\
Comptes fournisseurs & x & x\\
Charges à payer & x & x\\
Dividendes à payer & x & x\\
Emprunts à court terme & x & x\\
Obligations à court terme & x & x\\
Produits différés (we owe merch to someone) & x & x\\
\textbf{Total pasifs courants} & XX & XX\\
\hline
\uline{Passifs non courants} &  & \\
Emprunts à long terme & x & x\\
Hypothèque & x & x\\
\textbf{Total passifs non courants} & XX & XX\\
\hline
\uline{Capitaux propres} &  & \\
Capital social & x & x\\
Résultats non distribués & x & x\\
Surplus d'apport & x & x\\
\textbf{Total Capitaux Propres} & XX & XX\\
\hline
\textbf{Total passifs et capitaux propres} & XXX & XXX\\
\hline
\end{tabular}
\end{center}
\item Vérifications:
\label{sec:org56b8572}
Voir équation comptable A = P + CP
Les éléments ici influencent l'état des flux de trésorie qui a ses propres vérifications.
\item Remarques :
\label{sec:orge7999b9}
Document de synthèse qui expose à une date donnée la situation financière d'une
entreprise en fournissant un résumé de l'ensemble de ses éléments d'actif (A),
de passif(P)et de capitaux propres (C).

\begin{enumerate}
\item La situation financière se compose de deux parties principales: les
éléments d'actif et les sources d'actif (passif et capitaux propres).
\item Le total des éléments d'actif doit toujours égaler le total des sources
d'actif.
L'ÉQUATION COMPTABLE : A = P + C doit toujours être vérifiée.
\item À tout moment, il est possible d'établir les capitaux propres d'une
entreprise.
A - P = C
\end{enumerate}
\item Équation comptable
\label{sec:orgeb0fc68}
L'ÉQUATION COMPTABLE : A = P + C doit toujours être vérifiée.
\end{enumerate}

\item État des flux de trésorie (prochain cours)
\label{sec:orgde81623}
\begin{enumerate}
\item Tableau
\label{sec:orgc9ccbe7}
\begin{center}
\begin{tabular}{ll}
\hline
Compagnie YY & \\
État des flux de trésorie (méthode indirecte) & \\
Du 1er Janvier au 31 décembre 2018 & \\
\hline
Activités de opérationnelles & \\
\hline
+ Resultat net & x\\
+ Éléments sans effets sur la trésorie & x\\
(amortissement) & \\
(Dépréciation des comptes clients) & \\
+ Delta PC (Fin - Début) (exclure prov pour dépréc CC) & x\\
+ Delta AC (Début - Fin) (exclure encaisse et placements CT) & x\\
\hline
\textbf{Total Activités Opérationnelles} & XX\\
\hline
Activités de financement & \\
\hline
+ Émission d'actions & x\\
- Versement de dividendes & (x)\\
+ Nouvel emprunts & x\\
- Remboursement d'emprunts & (x)\\
- Frais financiers & (x)\\
\hline
\textbf{Total activités de financement} & XX\\
\hline
Activité d'investissement & \\
\hline
- Achat d'actifs non courant & (x)\\
+ Disposition d'actif non courant & x\\
\hline
\textbf{Total Activités d'Investissement} & XX\\
\hline
Variation de la trésorie & XXX\\
Trésorie de début & XX\\
Trésorie de fin & XX\\
\end{tabular}
\end{center}

\item Vérifications
\label{sec:orgec18a87}
\begin{quote}
\(\Delta_{\text{Tr}}\) = \(\Sigma_{\text{AO}}\) + \(\Sigma_{\text{AF}}\) + \(\Sigma_{\text{AI}}\)
= \(\Delta_{\text{Encaisse}}\) + \(\Delta_{\text{PlacementsCT}}\)

Tr\(_{\text{Début}}\) = Encaisse\(_{\text{Début}}\) + PlacementsCT\(_{\text{Début}}\)

Tr\(_{\text{Fin}}\) = Encaisse\(_{\text{Fin}}\) + PlacementsCT\(_{\text{Fin}}\)
\end{quote}
\end{enumerate}
\end{enumerate}
\end{enumerate}
\subsubsection{Cours 2 État des flux de trésorie \& Ratios financiers}
\label{sec:orgaf8da0c}
\begin{enumerate}
\item Etat de flux de trésorie mis avec les états financiers dans cours 1
\label{sec:org75b226c}
\item Ratios : Voir \href{https://moodle.polymtl.ca/pluginfile.php/512706/mod\_folder/content/0/Cours\%25202.1\%2520-\%2520Ratios.pdf?forcedownload=0}{Cours 2.1 Ratios.pdf}
\label{sec:orge4adedb}
\begin{enumerate}
\item Ratios de liquidité
\label{sec:org6bb99a6}
\begin{enumerate}
\item Ratio de liquidité courante (ratio du fond de roulement)
\label{sec:orga6d1c4f}
\begin{verbatim}
actif courants
---------------
passif courants
\end{verbatim}
\begin{enumerate}
\item Interprétation
\label{sec:orgca1e79c}
\textbf{Capacité d'une entreprise à faire face à ses obligations financières à court
terme lorsqu'elles viennent à échéance}
\begin{itemize}
\item < 1: la compagnie a de la misère à rembourser ses dettes
\item > 2: L'entreprise a trop de stocks
\item\relax [1.2, 2.0] : La majorité des analystes estiment que c'est idéal
\end{itemize}
\end{enumerate}
\item Ratio de liquidité relative (Ratio de liquidité immédiate)
\label{sec:org853b150}
\begin{verbatim}
actif courant - stocks - frais payés d'avance
---------------------------------------------
              passif courant
\end{verbatim}
\begin{enumerate}
\item Interprétation
\label{sec:org51d374f}
*Indique si on peut rembourser le passif à court terme sans avoir à vendre les
stocks. Si le ratio est nettement inférieur au ratio de liquidé courante, cela
signifie que l'actif à court terme dépend étroitement des stocks.*
\begin{itemize}
\item Idéal : [1,2]
\end{itemize}
\end{enumerate}
\end{enumerate}
\item Ratios d'endettement
\label{sec:orgbd2c2de}
\begin{enumerate}
\item Ratio d'endettement
\label{sec:orgbb03f09}
\begin{verbatim}
passif total
------------
actif total
\end{verbatim}
\begin{enumerate}
\item Interprétation
\label{sec:org7344ec4}
Capacité de respecter ses engagements à long terme
\begin{itemize}
\item < 30\% : Excellent
\item\relax [30\%, 36\%] : bon
\item > 40\% : Problématique
\end{itemize}
\end{enumerate}
\item Ratio de la couverture des intérêts
\label{sec:orgba87966}
\begin{verbatim}
Résultat avant intérêts et impôts
---------------------------------
       Intérêts
\end{verbatim}
\begin{enumerate}
\item Interprétation
\label{sec:orgd6a664c}
Indique dans quelle mesure les intérêts débiteurs sont couverts par les flux de
trésorie de la société.
\begin{itemize}
\item < 1 : La société peut éprouver de véritables difficultés à régler ses intérêts
débiteurs et le risque de défaut de paiement est jugé élevé.
\item > 1.5 : Idéal
\end{itemize}
\end{enumerate}
\end{enumerate}
\item Ratios de performance
\label{sec:org8712864}
\begin{enumerate}
\item Ratio de rotation de l'actif
\label{sec:orgd2f2863}
\begin{verbatim}
      Ventes nettes
-------------------------
Valeur moyenne de l'actif

Valeur moyenne = (montant de fin + montant de début) / 2
Ventes nettes = ventes moins les rendus, rabais et escomptes sur les ventes
\end{verbatim}
\begin{enumerate}
\item Interprétation
\label{sec:org7c99bde}
\begin{itemize}
\item Il indique le montant du chiffre d'affaires généré par chaque dollar investi
dans l'actif total.
\item Habituellement, plus le ratio est élevé, plus la gestion des actifs est
optimale.
\item Par exemple, un ratio de 0,60 \$ indique que chaque dollar investi rapporte
0,60 \$ de ventes nettes.
\item Ce ratio est utile pour se comparer à ses concurrents.
\end{itemize}
\end{enumerate}
\item Ratio de rotation des comptes clients
\label{sec:orgcc5890c}
\begin{verbatim}
         ventes nettes
----------------------------------
Valeur moyenne des comptes clients

Valeur moyenne = (montant de fin + montant de début) / 2
Ventes nettes = ventes moins les rendus, rabais et escomptes sur les ventes
\end{verbatim}
\begin{enumerate}
\item Interprétation
\label{sec:orgcfaaf9b}
\begin{itemize}
\item Il mesure le nombre de fois qu’une entreprise transforme ses créances clients en
ventes.
\item Plus ce ratio est faible, plus l’entreprise s’expose aux risques de non-paiement de la
part de ses partenaires, et plus elle devra revoir sa politique de crédit. Il s’agit donc,
pour l’entreprise, de trouver un juste milieu entre les créances accordées et les ventes
à générer via le crédit.
\item Un ratio de 15 signifie que l’entreprise collecte 15 fois ses compte-client par année. On
peut également conclure que ça lui prend en moyenne 24,3 jours pour collecter ses
clients (ratio du délai de recouvrement des c.c. 365/15).
\end{itemize}
\end{enumerate}
\item Ratio du délai de recouvrement des comptes clients
\label{sec:orgb957c84}
\begin{verbatim}
         365 jours
----------------------------
Rotation des Comptes Clients
\end{verbatim}
\begin{enumerate}
\item Interprétation
\label{sec:orgc8bb3f7}
\begin{itemize}
\item Mesure le nombre moyen de jours qu’il faut aux clients pour payer leurs comptes.
\item Donne une indication de l’efficacité des politiques de crédit et de recouvrement en
vigueur dans l’entreprise.
\item Permet de déterminer si les modalités de crédit dont se sert l’entreprise sont réalistes.
\item Pour déterminer si le délai moyen de recouvrement des comptes clients est adéquat, il
suffit de le comparer aux modalités de crédit que vous offrez à vos clients.
\begin{itemize}
\item Par exemple, si vous donnez à vos clients 30 jours pour payer leur facture et que
votre délai moyen de recouvrement des comptes clients est de 45 jours, il y a un
problème; en revanche, si le délai moyen est inférieur à 30 jours, c’est une bonne
nouvelle.
\end{itemize}
\end{itemize}
\end{enumerate}
\item Ratio de rotation stocks
\label{sec:org5f54452}
\begin{verbatim}
    Coûts des ventes
-------------------------
Valeur moyenne des stocks
\end{verbatim}
\begin{enumerate}
\item Interprétation
\label{sec:org0983b1f}
\begin{itemize}
\item Un faible ratio de rotation est généralement de mauvais augure, car les produits ont
tendance à s'abîmer s'ils restent trop longtemps dans un entrepôt.
\item Les sociétés qui vendent des biens périssables ont un ratio de rotation très élevé.
\item Ce ratio est utile pour se comparer à ses concurrents.
\item Voici un lien utile afin de pouvoir comparer le ratio de votre entreprise à celui de votre
secteur d’activité :
\url{https://www.bdc.ca/fr/articles-outils/boite-outils-entrepreneur/evaluation-entreprise/pages/rotation-stocks-outil-analyse-comparative.aspx?ChangeIndustry=1}
\end{itemize}
\end{enumerate}
\item Ratio de rotation de l'actif immobilisé
\label{sec:orgdd70e4b}
\begin{verbatim}
            ventes nettes
--------------------------------------
valeur moyenne des actifs non courants

Valeur moyenne = (montant de fin + montant de début) / 2
Ventes nettes = ventes moins les rendus, rabais et escomptes sur les ventes
\end{verbatim}
\begin{enumerate}
\item Interprétation
\label{sec:org9a6cf65}
\begin{itemize}
\item Le ratio de rotation des immobilisations (rotation de l’actif immobilisé)
indique combien de revenus génère chaque dollar investi dans les
immobilisations.
\item Habituellement, plus le ratio est élevé, plus la gestion des actifs
immobilisés est optimale.
\item Attention à l’interprétation du ratio: Une entreprise bien installée possédant
de vieux équipements fortement amortie aurait un meilleur résultat en
opposition à une plus jeune entreprise possédant de nouveaux équipements.
\item Il faut donc comparer des entreprises possédant des immobilisations d’âge
comparable.
\end{itemize}
\end{enumerate}
\end{enumerate}
\item Ratios de rentabilité (rendement)
\label{sec:org5e28a28}
\begin{enumerate}
\item Ratio de la marge nette
\label{sec:orgc26724f}
\begin{verbatim}
Résultats nets
--------------
Ventes nettes

Ventes nettes : ventes moins les rendus, rabais et escomptes sur les ventes
Résultat net : résultat brut moins les charges
Résultat brut : ventes moins le coût des vente
\end{verbatim}
\begin{enumerate}
\item Interprétation
\label{sec:org16278c2}
Rentabilité de l'exploitation
Ce ratio indique la partie des ventes qui contribue au bénéfice de l'entreprise.
\begin{itemize}
\item Le ratio ne sert à rien si l'entreprise perd de l'argent, car elle ne fait
alors pas de profit.
\item Lorsque le ratio de marge bénéficiaire nette (marge nette) est faible, cela
peut être attribuable à la stratégie prix et/ou à l'incidence que la
concurrence a sur la marge.
\item Une marge élevée est un bon signe
\item Ce ratio est utile pour se comparer à ses concurrents.
\end{itemize}
\end{enumerate}
\item Ratio de la marge brute
\label{sec:orgbc4a7e3}
\begin{verbatim}
Resultats bruts
---------------
 Ventes nettes
\end{verbatim}
\begin{enumerate}
\item Interprétation
\label{sec:orgfa19ade}
Indique le niveau de bénéfice généré par l'entreprise.

Ce ratio témoigne de la politique de fixation des prix de la société et de la majoration
réelle des prix.

Par exemple, si le ration de la marge bénéficiaire brute (marge brute) est de 33 \%,
cela signifie que le prix des produits est majoré de 50 \% (1/(1-33 \%))-100 \%).
\begin{itemize}
\item Les résultats peuvent être biaisés si l'éventail de produits de la société est
très large.
\item Ce ratio se révèle très utile lorsqu'il est comparé à ceux des exercices
antérieurs.
\item En général, la marge bénéficiaire brute doit être stable. Elle ne doit pas
fluctuer d'une période à l'autre.
\end{itemize}
\end{enumerate}
\item Ratio de rentabilité
\label{sec:org7dda9b4}
\begin{verbatim}
         résultats nets
-----------------------------------
valeur moyenne des capitaux propres

Valeur moyenne = (Montant de fin + montant début)/2
Résultat net = résultat brut moins les charges
\end{verbatim}
\begin{enumerate}
\item Interprétation
\label{sec:org9ba2d61}
Ce ratio indique le taux de rendement que l'entreprise tire de l'investissement
de ses propriétaires ou que les actionnaires obtiennent de leurs actions.

Par exemple, si le ratio s'élève à 10 \%, cela signifie que chaque dollar investi
à l'origine produit un actif de 10 cents.

\begin{itemize}
\item Les sociétés à forte croissance devraient obtenir un rendement élevé des capitaux
propres.
\item Le rendement moyen des capitaux propres au cours des 5 à 10 dernières années
donne une meilleure idée de la croissance à long terme.
\end{itemize}
\end{enumerate}
\end{enumerate}

\item Ratios de la valeur marchande
\label{sec:org7cada24}
\begin{enumerate}
\item Ratio du résultat par action
\label{sec:org5b8877f}
\begin{verbatim}
            Résultats nets
------------------------------------------
Nombre d'actions ordinaires en circulation

Résultats net = résultat brut moins les charges
\end{verbatim}
\begin{enumerate}
\item Interprétation
\label{sec:org6dcbab2}
C'est le ratio le plus utilisé. Il indique le montant du bénéfice généré par la
société, par action.
\begin{itemize}
\item On compare ce ratio avec les exercices précédents ainsi qu’avec d’autres
entreprises.
\item Lorsque la société émet de nouvelles actions, il est beaucoup plus difficile
de comparer le BPA de l'exercice en cours et des exercices précédents.
\item Le ratio du bénéfice par action (BPA ou du résultat par action) est
principalement utilisé pour les sociétés cotées en bourse. En soi, le BPA
n'indique pas grand-chose. Par contre, si vous le comparez au BPA d'un
trimestre ou d'un exercice antérieur, vous pouvez calculer le taux de
croissance du bénéfice (par action) de la société. Par exemple, une entreprise
qui augmente de 50 \% en un an possède un excellent taux de croissance.
\end{itemize}
\end{enumerate}
\item Ratio du cours/résultats
\label{sec:org9423563}
\begin{verbatim}
Cours du marché ordinaire
-------------------------
Résultat par action

Résultats net = résultat brut moins les charges
\end{verbatim}
\begin{enumerate}
\item Interprétation
\label{sec:org832d235}
L'un des ratios les plus utilisés, il permet de comparer le prix courant au
bénéfice afin de déterminer si le titre est surévalué ou sous-évalué. Il
représente une attente quant au rendement futur de l'entreprise.
\begin{itemize}
\item En règle générale, lorsque le ratio est élevé, cela signifie que les
investisseurs prévoient une forte croissance dans le futur.
\item Le ratio cours/bénéfice (cours/résultats, C/B) moyen du marché est de 20 à 25
fois le bénéfice.
\item Les sociétés qui perdent de l'argent n'ont pas de ratio C/B.
\item On compare ce ratio avec les exercices précédents ainsi qu’avec d’autres
entreprises.
\item Le ratio C/B ne détermine pas le cours de l'action. Un ratio C/B bas peut
indiquer que le bénéfice de l'entreprise est stable ou qu'il croît lentement,
comme il peut signifier que l'entreprise éprouve des difficultés financières.
\end{itemize}
\end{enumerate}
\end{enumerate}
\end{enumerate}
\end{enumerate}
\subsubsection{Cours 3 Immobilisations et amortissement}
\label{sec:orga7e4d41}
\begin{enumerate}
\item Vocabulaire
\label{sec:orgcc96679}
\textbf{Actif immobilisé/Immobilisations} : 
\begin{itemize}
\item Destinés à être utilisés pour la production de bien ou pour gagner du revenu.
\item Destinés à être utilisés de façon durable
\item Non destinés à être vendus
\end{itemize}
\begin{enumerate}
\item Catégories d'immobilisations
\label{sec:org7bab5cd}
\begin{enumerate}
\item Immobilisations corporelles
\label{sec:orgb7cdb30}
\begin{itemize}
\item Biens amortissables : 
\begin{itemize}
\item Immeubles
\item Biens qui s'épuisent
\end{itemize}
\item Biens non-ammortissables : 
\begin{itemize}
\item Terrain
\end{itemize}
\end{itemize}
\item Immobilisations incorporelles
\label{sec:org348ac16}
\begin{itemize}
\item R\&D
\item Droits d'auteur
\item Franchises
\item Licenses
\item Marque de commerce
\item Autres propriétés intellectuelles
\end{itemize}
\end{enumerate}
\end{enumerate}

\item Coût d'un actif immobilisé
\label{sec:org79531bf}
L'idée est de prendre en compte 
\begin{center}
\begin{tabular}{l}
Coût de l'actif (prix affiché)\\
+ Coûts de mise en service\\
+ Autres frais comme transport, préparation, installation\\
\end{tabular}
\end{center}


\item Amortissement
\label{sec:org7b83484}
\begin{enumerate}
\item Amortissement linéaire
\label{sec:orgca2418b}
L'immobilisation passe du coût initial \textbf{P} à sa valeur résiduelle \textbf{R} en \textbf{n}
années en perdant \textbf{(P-R)/n} à chaque année (sauf la première année où on prend
une fraction de ce montant correspondant à la fraction d'année d'utilisation(m/12)).

D\(_{\text{1}}\) = (m/12) * (P-R)/n

D\(_{\text{t}}\) = (P-R)/n

\item Amortissement à taux dégressif à taux constant
\label{sec:orge8842d7}

Aller de \textbf{P} à \textbf{R} en \textbf{n} étapes en multipliant par un facteur.

L'idée est en fait de multiplier la valeur de l'année précédente par un facteur
de sorte qu'après avoir multiplié par ce facteur \textbf{n} fois, on fini par avoir
multiplié par \textbf{R/P}.

\begin{center}
\begin{tabular}{rl}
Année & Valeur comptable\\
\hline
0 & P\\
1 & P (R/P)\(^{\text{1/n}}\)\\
2 & P (R/P)\(^{\text{1/n}}\) (R/P)\(^{\text{1/n}}\)\\
\dots{} & \dots{}\\
n & P ( (R/P)\(^{\text{1/n}}\) )\(^{\text{n}}\) = P (R/P) = R\\
\end{tabular}
\end{center}


En pratique, on utilise la terminologie suivante:
On calcule un taux basé sur \textbf{P}, \textbf{R} et \textbf{n}:

d = 1 - (R/P)\(^{\text{1/n}}\)

et au lieu de soustraire le même montant à chaque année, on ammortit de

D\(_{\text{t}}\) = CNA\(_{\text{t-1}}\) * d

Ce qui correspond à multiplier par (R/P)\(^{\text{1/n}}\):

CNA\(_{\text{t}}\) = CNA\(_{\text{t-1}}\) - CNA\(_{\text{t-1}}\) * d = CNA\(_{\text{t-1}}\) * (1 - d) = CNA\(_{\text{t-1}}\) * (R/P)\(^{\text{1/n}}\)

\item Amortissement Proportionnel à l'ordre inversé des années
\label{sec:org9c99146}

Soit U\(_{\text{t}}\) = (P-R)(n - t + 1)/k la perte de valeur dans la première année \textbf{d'utilisation}.

Si les années d'utilisation sont désynchronisées avec les périodes d'éxercice,
chaque année d'exercice ira chercher une partie de chaque année d'utilisation qui
la chevauchent.

Par exemple, pour une machinne achetée en fin avril de l'année 1: on a
U\(_{\text{1}}\) = (P-R)(n-1+1)/k, mais seulement (8/12)U\(_{\text{1}}\) sera mis dans les livres pour
l'année 1.

D\(_{\text{1}}\) = (8/12)U\(_{\text{1}}\)

À l'année fiscale 2, on a 4 mois qui font partie de la première année
d'utilisation (U\(_{\text{1}}\)).  L'amortissement à l'année 2 incluera (4/12)D\(_{\text{1}}\) et aussi (8/12)D\(_{\text{2}}\)
où

D\(_{\text{2}}\) = (4/12)U\(_{\text{1}}\) + (8/12)U\(_{\text{2}}\)

\textbf{En passant, k = n(n+1)/2.}

/Sti que ça vaut pas la peine d'apprendre cet esti d'amortissement, come on.
R'garde, si j'ai besoin de faire un amortissement comme ça, j'ouvrirai un
livre./


\item Amortissement Proportionnel à l'utilisation
\label{sec:org2613e5d}

Ben simple: on remplace le \textbf{n} par un volume total d'utilisation \textbf{V}.

on a donc d = (P-R)/V

L'amortissement à l'année \textbf{t} sera donné par le volume d'utilisation de v\(_{\text{t}}\) de
l'année \textbf{t}.

D\(_{\text{t}}\) = d*v\(_{\text{t}}\)

Par exemple, ça pourrait être une voiture, \textbf{V} serait le nombre total de
kilomètres qu'on prévoit faire avec la voiture et v\(_{\text{t}}\) serait le nombre de
kilomètres faits durant l'année \textbf{t}.
\end{enumerate}
\end{enumerate}

\subsubsection{Cours 4 Intérêt et valeur de l'argent dans le temps}
\label{sec:org04ea299}
À cause des intérêts ou de l'inflation, un montant X\(_{\text{1}}\) à un temps T\(_{\text{1}}\) n'aura pas
la même valeur à un temps T\(_{\text{2}}\).  Les méthodes dans cette sections permettent de
trouver des équivalences entre des montants d'argents à différents temps.

Les facteurs d'actualisations permettent d'établire ces équivalences.

Les taux effectifs permettent de comparer des situations avec des fréquences de
capitalisations différentes.
\begin{enumerate}
\item Valeur de l'argent dans le temps
\label{sec:org8b1827d}

\begin{enumerate}
\item La base (P/F;i;n), (F/P;i;n)
\label{sec:org2ab2330}

Avec un taux i, un montant est multiplié par (1+i) à chaque période de
capitalisation.

Ainsi, un montant à l'année 0 aura été mutiplié par (1+i)\(^{\text{n}}\) rendu à l'année n.

Ainsi, si on connait le montant à l'année n, on peut connaître le montant à
l'année 0 en multipliant par (1+i)\(^{\text{-n}}\).  Donc:

\begin{center}
\begin{tabular}{ll}
(P/F, i, n) & (1+i)\(^{\text{-n}}\)\\
(F/P, i, n) & (1+i)\(^{\text{n}}\)\\
\end{tabular}
\end{center}

\item Le reste
\label{sec:org1b55ef8}

Avec des annuités, on dépose un montant A à chaque période. Au bout de n années,
le premier montant A déposé à l'année vaudrait A(1+i)\(^{\text{-1}}\) à l'année 0, le montant
déposé à l'année 2 vaudrait A(1+i)\(^{\text{-2}}\) à l'année 0.  Finalement le montant déposé
à l'année n vaudrait A(1+i)\(^{\text{-n}}\) à l'année 0.

La somme est donc

A((1+i)\(^{\text{-1}}\) + (1+i)\(^{\text{-2}}\) + \dots{} + (1+i)\(^{\text{-n}}\)) = P

Si on fait le même truc qu'avec une série géométrique pour trouver P/A = \(\xi\), nous
avons

\(\xi\) = (1+i)\(^{\text{-1}}\) + (1+i)\(^{\text{-2}}\) + \dots{} + (1+i)\(^{\text{-n}}\)

et

(1+i)\(\xi\) = 1 + (1+i)\(^{\text{-1}}\) + (1+i)\(^{\text{-2}}\) + \dots{} + (1+i)\(^{\text{-(n-1)}}\) 

donc \(\xi\) - (1+i)\(\xi\) = -i\(\xi\) = (1+i)\(^{\text{-n}}\) - 1.

ou bien i\(\xi\) = 1 - (1+i)\(^{\text{-n}}\) et donc \(\xi\) = (P/A;i;n) = (1 - (1+i)\(^{\text{-n}}\))/i

On peut multiplier par (1+i)\(^{\text{n}}\)/(1+i)\(^{\text{n}}\) = 1 pour obtenir

\begin{center}
\begin{tabular}{ll}
(P/A, i, n) & ((1+i)\(^{\text{n}}\) - 1)/(i(1+i)\(^{\text{n}}\))\\
(A/P, i, n) & 1/(P/A, i, n)\\
\end{tabular}
\end{center}

pour avoir une formule sans exposants négatifs.
(setq org-pretty-entities nil)

Les autres formules peuvent être obtenues ainsi:

\begin{center}
\begin{tabular}{ll}
(F/A, i, n) & ((1+i)\(^{\text{n}}\) - 1)/i\\
(A/F, i, n) & i / ((1+i)\(^{\text{n}}\) - 1)\\
(P/G, i, n) & \ldots{}\\
\end{tabular}
\end{center}

\item A\(_{\infty}\)
\label{sec:orgececdb2}
On veut que l'intérêt annuel nous donne un montant A\(_{\infty}\) à chaque année.

P * (1 + i) = P + A\(_{\infty}\) 

et on retire le A\(_{\infty}\) pour faire quelque chose avec, et on s'en fait donner un
autre l'année d'après jusqu'à la fin des temps.

Donc A\(_{\infty}\) = i * P \(\Leftrightarrow\) P = A\(_{\infty}\) / i

Le P dans P = A\(_{\infty}\) / i est appelé le coût immobilisé

\item Règle du 72
\label{sec:org3bfc61c}

n = 0.72 / i \textasciitilde{} nombre d'années pour doubler au taux i
\end{enumerate}
\end{enumerate}


\subsubsection{Cours 5 Taux d'intérêts nominal et réel}
\label{sec:orge56a3d8}

Formules pour convertir entre différens taux d'intérêts équivalents et exprimés.

\begin{enumerate}
\item Taux nominal et périodes de capitalisation
\label{sec:org42b20bd}

Les termes d'un prêt ou d'un investissement expriment les taux sous la forme

"Un taux nominal de i\(_{\text{nominal}}\) par année capitalisée m fois par année"

Ceci veux dire que m fois par année, le montant sera multiplié par (1 + i/m).
Au bout d'une année, le montant aura été multiplié par (1 + i/m)\(^{\text{m}}\).

On appelle i\(_{\text{eff}}\) = (1 + i\(_{\text{nom}}\)/m)\(^{\text{m}}\).

Afin de pouvoir travailler avec des flux monétaires avec diverses fréquences de
capitalisation, nous devons transformer ces taux en taux effectifs de la même fréquence.

\item Comparaisons
\label{sec:org9c7e8fa}

(1 + i\(_{\text{1}}\)/m)\(^{\text{m}}\) = (1 + i\(_{\text{2}}\)/\(\nu\))\^{}\(\nu\)

(1 + i\(_{\text{1}}\)/m)\(^{\text{m/}\nu}\) = 1 + i\(_{\text{2}}\)/\(\nu\)

i\(_{\text{2}}\)/\(\nu\) = (1 + i\(_{\text{1}}\)/m)\(^{\text{m/}\nu}\) - 1
\end{enumerate}


\subsubsection{Cours 6 Seuil de rentabilité = Point Mort et analyse marginale}
\label{sec:org7fa4ad1}
\begin{enumerate}
\item Contexte
\label{sec:orgd34c9d3}
On suppose que les ventes et côuts suivent une relation

Ventes = PV\(_{\text{u}}\) * U

Coûts = CV\(_{\text{u}}\) * U + CF

Coûts variables totaux = CV\(_{\text{u}}\) * U

où U est le nombre d'unités vendues d'un produit, PV\(_{\text{u}}\), CV\(_{\text{u}}\) et CF sont des
constantes en \$/u, \$/u et \$ respectivement.

\item Point Mort (Nombre d'unités qui rend les ventes égales aux coûts)
\label{sec:org3e67ff8}

Le point mort U\(^{\text{*}}\) est la solution de l'équation des recettes et des coûts.

\begin{quote}
Pv * U\(^{\text{*}}\) = CT(U\(^{\text{*}}\)) = CV\(_{\text{u}}\) * U\(^{\text{*}}\) + CF
\end{quote}

Ça nous donne un nombre d'unités:

\(\Leftrightarrow\) (PV - CV\(_{\text{u}}\)) * U\(^{\text{*}}\) = CF

\(\Leftrightarrow\) U\(^{\text{*}}\) = CF / (PV - CV\(_{\text{u}}\))

\item Différentes expressions du point mort
\label{sec:org97fd66d}

SR(Q) = U\(^{\text{*}}\) = CF / (PV\(_{\text{u}}\) - CV\(_{\text{u}}\)) = CF / CM\(_{\text{u}}\) \(\Leftrightarrow\) CF = U\(^{\text{*}}\) * CM\(_{\text{u}}\)

SR(\$) = U\(^{\text{*}}\) * PV\(_{\text{u}}\) = (U\(^{\text{*}}\) * CM\(_{\text{u}}\)) * (PV\(_{\text{u}}\)/CM\(_{\text{u}}\)) = CF * 1/CM\_\%

\item Contributions marginales
\label{sec:org93db10a}

CM\(_{\text{u}}\) = PV\(_{\text{u}}\) - CV\(_{\text{u}}\) contribution marginale unitaire.

CM\_\% = CM\(_{\text{u}}\) / PV\(_{\text{u}}\) marginale en pourcentage

CM\_\$(U) = Ventes - Coûts Variables Totaux = PV\(_{\text{u}}\) * U - CV\(_{\text{u}}\) * U : (contribution marginale totale)

C'est le profit qu'on ferait si les couts fixes disparaissaient.

NOTE: CM\_\$(U\(^{\text{*}}\)) = CF

\item Méthode des points extrêmes
\label{sec:org93cd2ad}

On se fait donner un tableau avec des lignes (x,y) = (quantité, couts).

On suppose que ces coûts suivent une règle y = ax + b avec a = CV\(_{\text{u}}\) et b = CF.

On choisi deux points (x\(_{\text{1}}\), y\(_{\text{1}}\)), (x\(_{\text{2}}\), y\(_{\text{2}}\)) éloignés dans le tableau:

a = (y\(_{\text{2}}\) - y\(_{\text{1}}\)) / (x\(_{\text{2}}\) - x\(_{\text{1}}\))

maintenant qu'on a calculé a, on peut utiliser un des points pour calculer b:

y\(_{\text{1}}\) = a x\(_{\text{1}}\) + b \(\Leftrightarrow\) b = y\(_{\text{1}}\) - a x\(_{\text{1}}\)

\item Point d'équivalence de deux projets
\label{sec:org5f8d42f}

Ah point d'équivalence entre deux options pour produire la même chose. Soit deux
options données par CV\(_{\text{u1}}\), CF\(_{\text{1}}\) et CV\(_{\text{u2}}\), CF\(_{\text{2}}\).  Alors CV\(_{\text{U1}}\) * U\(_{\text{eq}}\) + CF\(_{\text{1}}\) = CV\(_{\text{U2}}\) *
U\(_{\text{eq}}\) + CF\(_{\text{2}}\) nous donne

PE(Q) = U\(_{\text{eq}}\) = (CF\(_{\text{2}}\) - CF\(_{\text{1}}\)) / (CV\(_{\text{u1}}\) - CV\(_{\text{u2}}\))

CM\_\%1 * PE(\() - CF_1 = CM_%2 * PE(\)) - CF\(_{\text{2}}\)

PE(\$) = (CF\(_{\text{1}}\) - CF\(_{\text{2}}\)) / (CM\_\%2 - CM\_\%1)

\item Marge de sécurité
\label{sec:org67f6c1a}

Soit U un nombre d'unités

MS(\$) = RevenusTotauxPrévus - RevenusAuSeulDeRentabilité = PV\(_{\text{u}}\) * U - PV\(_{\text{u}}\) * U\(^{\text{*}}\)

MS(Q) = U - U\(^{\text{*}}\) = (nombre d'unités vendues de plus que le point mort)

MS(\%) = MS(\$)/(PV\(_{\text{u}}\) * U) = PV\(_{\text{u}}\)(U-U\(^{\text{*}}\))/(PV\(_{\text{u}}\)*U) = 

\item Profit/Perte par unité
\label{sec:orgc7f0d54}

Profit(U) = PV\(_{\text{u}}\) * U - (CV\(_{\text{u}}\) * U + CF)

Profit(U) / U = ProfitParUnité

= (PV\(_{\text{u}}\) * U - (CV\(_{\text{u}}\) * U + CF)) / u

= ((PV\(_{\text{u}}\) - CV\(_{\text{u}}\)) * U - CF) / U

= (PV\(_{\text{u}}\) - CV\(_{\text{u}}\)) - CF/U



\item Vocabulaire
\label{sec:orgb0a93ef}
\begin{enumerate}
\item Composantes du coût de fabrication (ou de production)
\label{sec:org747c1e7}

\begin{itemize}
\item Matières premières (MP) \\
Note: C'est un coût variable
\item Main d'oeuvre directe (MOD) \\
Salaires de ceux qui travaillent \emph{directement} à la transofrmation des
matières premières en produits finis.
\item Frais généraux de fabrication (FGF) autres que la \emph{Main d'oeuvre directe} et les
\emph{matières premières}.
\end{itemize}

\item Frais généraux de fabrication
\label{sec:org55cd433}

\textbf{Definition} : Tous les autres coûts que main d'oeuvre directe (MOD) et la
matière première (MP).

\textbf{Exemples} : 
\begin{itemize}
\item Amortissements
\item Chauffage et éclairage
\item Entretien des équipements
\item Assurances, impôt foncier
\item Main-d'oeuvre indirecte: 
\begin{itemize}
\item Salaires des contremaîtres
\item employés de bureau,
\item ingénieurs
\item programmeurs
\end{itemize}
\end{itemize}

\textbf{Notes} : Certains FGF peuvent être des coûts fixes et d'autres peuvent être des
coûts variables.

Les FGF sont répartis aux produits.

\item Coûts pertinents
\label{sec:orgf5db055}

\begin{itemize}
\item Coûts différentiels et coûts stables
\item Coûts engagés et coûts d'opportunité
\item Coûts passés
\end{itemize}
\end{enumerate}

\item État des flux de trésorie différentiel
\label{sec:org3772cd0}
Voir \url{https://moodle.polymtl.ca/pluginfile.php/163307/mod\_resource/content/35/SSH3201\%20Cours\%206\%20Couts\_PM\_A2018\_modifié\_PDF.pdf}
\end{enumerate}
\subsubsection{Cours 7 VAN, IR, DR, TRI, TRIM}
\label{sec:orged21d40}

\begin{enumerate}
\item VAN : Valeur Actualisée Nette
\label{sec:org5057e0b}

C'est la somme des valeurs actualisées de tous les flux monétaires d'un projet.

La VAN est fonction du taux utilisé pour les actualisations.  On peut donc
généraliser la définition:

VAN(i) : La somme des flux monétaires actualisés au taux i du projet

On utilise généralement i = TRAM en l'absence d'indication contraire.

\item TRI : Le taux
\label{sec:org63da956}

Le TRI est la solution i\(^{\text{*}}\) de l'équation VAN(i\(^{\text{*}}\)) = 0.

Soient V\(_{\text{1}}\) = VAN(i\(_{\text{1}}\)), V\(_{\text{2}}\) = VAN(i\(_{\text{2}}\)), V\(^{\text{*}}\) = VAN(i\(^{\text{*}}\)) = 0.

\item Calcul du TRI par interpolation
\label{sec:org5554cf8}

Alors par un argument de triangles semblables, on a

(i\(^{\text{*}}\) - i\(_{\text{1}}\))/(V\(^{\text{*}}\) - V\(_{\text{1}}\)) = (i\(_{\text{2}}\) - i\(_{\text{1}}\))/(V\(_{\text{2}}\) - V\(_{\text{1}}\))

Ce qui nous permet d'isoler i\(^{\text{*}}\):

\(\Leftrightarrow\) i\(^{\text{*}}\) - i\(_{\text{1}}\) = (V\(^{\text{*}}\) - V\(_{\text{1}}\)) * (i\(_{\text{2}}\) - i\(_{\text{1}}\))/(V\(_{\text{2}}\) - V\(_{\text{1}}\))

\(\Leftrightarrow\) i\(^{\text{*}}\) =  i\(_{\text{1}}\) + (V\(^{\text{*}}\) - V\(_{\text{1}}\)) * (i\(_{\text{2}}\) - i\(_{\text{1}}\))/(V\(_{\text{2}}\) - V\(_{\text{1}}\))

\(\Leftrightarrow\) i\(^{\text{*}}\) =  i\(_{\text{1}}\) - V\(_{\text{1}}\) * (i\(_{\text{2}}\) - i\(_{\text{1}}\))/(V\(_{\text{2}}\) - V\(_{\text{1}}\)),  (parce que V\(^{\text{*}}\) = 0)

Et si on veut la formule donnée dans certains powerpoints, on a jsute à changer
le + pour un - et inverser l'ordre de V\(_{\text{1}}\), V\(_{\text{2}}\) dans le dénominateur.

\begin{quote}
\(\Leftrightarrow\) i\(^{\text{*}}\) =  i\(_{\text{1}}\) + V\(_{\text{1}}\) * (i\(_{\text{2}}\) - i\(_{\text{1}}\))/(V\(_{\text{1}}\) - V\(_{\text{2}}\)) \uline{FF}
\end{quote}

\item TRIM
\label{sec:orgaa4841b}

MP = Somme des \textbf{valeurs actuelles} des flux monétaires nets négatifs

MF = Somme des \textbf{valeurs futures} des flux monétaires positifs

MP = MF(1 + TRIM)\(^{\text{-n}}\)

MP/MF = (1 + TRIM)\(^{\text{-n}}\)

MF/MP = (1 + TRIM)\(^{\text{n}}\)

(MF/MP)\(^{\text{1/n}}\) - 1 = TRIM

\item Indice de rentabilité
\label{sec:orge903fc0}

IR = (VA des Flux Monétaires) / (VA des investissements) \\
Parce que VAN = VA\(_{\text{Recettes}}\) - |VA\(_{\text{INV}}\)| (ce qui rentre - ce qui sort) \\
donc VA\(_{\text{Recettes}}\) / |VA\(_{\text{INV}}\)| = (VAN + |VA\(_{\text{INV}}\)|) / |VA\(_{\text{INV}}\)| = VAN/|VA\(_{\text{INV}}\)| + 1 \\
\item Comparaison de projets de durées différentes
\label{sec:org5b6c57c}

Ramener à un m
Critére du service égales
PPCM

Période d'étude

\item Annuité Équivalente
\label{sec:orgf621e81}

Étant donné un projet avec des flux monétaires FMN\(_{\text{t}}\) et une durée de vie n, et un
TRAM i,

AE = VAN * (A/P;i;n)

\item Charges d'exploitation annuelle
\label{sec:org7a08d36}

CEA = (\(\Sigma_{\text{t}}\) CH\(_{\text{t}}\)(P/F;i;t) ) * (A/P;i;n)

Transformer les coûts en annuité équivalente

\item Cout annuel équivalent
\label{sec:orgcc7bc53}

\begin{enumerate}
\item DI = somme des valeurs actualisées des investissements \\
\item VAC = somme des valeurs actualisées des coûts \\
\item VAR = somme des valeurs actualisées des valeurs de revente \\
\end{enumerate}

CAE = -(|DI| + |VAC| - |VAR|) * (A/P;i;n)

Attention aux signes, puisque DI < 0, VAC < 0, VAR > 0 en tant que flux
monétaires, CAE = (DI + VAC + VAR) * (A/P;i;n) mais les valeurs absolues sont là
pour qu' on voit quels doivent être les signes des affaires.

\item Recouvrement du capital
\label{sec:org6c781c6}

RC = (|DI| - |VAR|) * (A/P;i;n)

\item Délai de recouvrement
\label{sec:org49a287f}

\begin{enumerate}
\item Input: Un tableau
\label{sec:org04f6927}

\begin{center}
\begin{tabular}{rl}
Année & FMN\\
0 & FMN\(_{\text{0}}\)\\
1 & FMN\(_{\text{1}}\)\\
2 & FMN\(_{\text{2}}\)\\
3 & FMN\(_{\text{3}}\)\\
\end{tabular}
\end{center}

\item Output: Les cumuls
\label{sec:orgc37f937}

\begin{center}
\begin{tabular}{rll}
Année & FMN & CUMUL\\
0 & FMN\(_{\text{0}}\) & FMN\(_{\text{0}}\)\\
1 & FMN\(_{\text{1}}\) & CUMUL\(_{\text{0}}\) + FMN\(_{\text{1}}\)\\
2 & FMN\(_{\text{2}}\) & CUMUL\(_{\text{1}}\) + FMN\(_{\text{2}}\)\\
3 & FMN\(_{\text{3}}\) & CUMUL\(_{\text{2}}\) + FMN\(_{\text{3}}\)\\
\end{tabular}
\end{center}

On trouve plus petit \textbf{n} tel que CUMUL\(_{\text{n}}\) < 0 < CUMUL\(_{\text{n+1}}\)

\item Interpolation pour le nombre de jours
\label{sec:org05baa25}
Ensuite, on interpole sur une droite passant par (0, cumul\(_{\text{n}}\)) et (1, cumul\(_{\text{n+1}}\))

cumul(k) = (cumul\(_{\text{n+1}}\) - cumul\(_{\text{n}}\)) * k + cumul\(_{\text{n}}\)

juste pour être sur :\\
cumul(0) = cumul\(_{\text{n}}\) \\
cumul(1) = cumul\(_{\text{n+1}}\) - cumul\(_{\text{n}}\) + cumul\(_{\text{n}}\) = cumul\(_{\text{n+1}}\) \\

cumul(\(\xi\)) = 0 \(\Leftrightarrow\) \(\xi\) = cumul\(_{\text{n}}\) / (cumul\(_{\text{n+1}}\) - cumul\(_{\text{n}}\)) = cumul\(_{\text{n}}\) / FMN\(_{\text{n+1}}\)

On donne la réponse sous la forme suivante: Le délai de recouvrement du projet
donné par les flux monétaires FMN\(_{\text{i}}\) a un délai de recouvrement 

\item Valeur
\label{sec:orgebd181a}

\begin{quote}
DR = n années et 365\(\cdot \xi\) jours
\end{quote}

Le fait que cumul\(_{\text{n}}\) et cumul\(_{\text{n+1}}\) soient de signes différents garanti que x \(\in\) [0,1].

\item Conclusions
\label{sec:org7df16a3}

On peut comparer le DR au délai maximal fixé par les gestionnaires et décider de
faire ou ne pas faire le projet
\end{enumerate}
\end{enumerate}
\subsubsection{Cours 8(Anas)}
\label{sec:org5d70f34}
\begin{enumerate}
\item + ANalyse marginale
\label{sec:org3dfcf47}

Pourcentage des bénéfices = BénéficeNet
\end{enumerate}
\subsubsection{Cours 8 Métriques de comparaison (AE, CAE, RC, DUE), remplacement}
\label{sec:org0a2d205}
\subsubsection{Cours 9,10 : Impôt}
\label{sec:org79166b7}

\begin{enumerate}
\item Taux fiscal effectif
\label{sec:org72e107b}

Impôt à Payer = Revenu Imposable * T

\item Règle de mise en service et de la demie année
\label{sec:org31aeece}

\begin{enumerate}
\item Règle de mise en service
\label{sec:org1dff2ee}

Bien autre qu'un batiment : date de première utilisiation
Bâtiment : date à partir de laquelle 90\% ou plus du bâtiment est utilisé
Bâtiment que je construit : date de 
\begin{itemize}
\item fin de construction, reno ou modification ou
\item 90\% est utilisé
\end{itemize}

\item Règle de la demie année
\label{sec:orgcbc7825}

Chaque actif est amorti comme s'il avait été mis en service au milieu de son
année d'acquisition.  On verra l'effet de ça dans les formules de déductions
pour amortissement.

\item Déductions pour amortissement (dégressif)
\label{sec:org048d1a4}
DPA\(_{\text{1}}\) = (DI * d) / 2 \\
FNACC\(_{\text{1}}\) = DI - DPA\(_{\text{1}}\)

FNACC\(_{\text{n}}\) = FNACC\(_{\text{1}}\) * (1-d)\(^{\text{n-1}}\) = DI * (1 - d/2) * (1-d)\(^{\text{n-1}}\) \\
(On multiplie n fois par quelque chose, (1-d/2) la première fois et (1-d) toutes
les autres fois)

DPA\(_{\text{n}}\) = d * FNACC\(_{\text{n-1 }}\)\\
= d * DI * (1 - d/2) * (1 - d)\(^{\text{n-2}}\)

\item Revenu imposable
\label{sec:org9dde0d5}

Revenu imposable = BénéficeNet - DPA

\item Impôt
\label{sec:org256c164}

Impot = RevenuImposable * T

\item BénéficeNetAprèsImpôt
\label{sec:org839ac67}

BN après Impôt = BénéviceNet - Impôt
\end{enumerate}
\end{enumerate}





\subsubsection{Cours 11,12 : Risque}
\label{sec:orgae14b1d}
\subsection{Don't forget:}
\label{sec:orgef9dc84}

\begin{enumerate}
\item Remplacer ou pas un actif
\label{sec:org348bb63}
\end{enumerate}
\subsection{Matière}
\label{sec:org1d7244b}

\subsubsection{États financiers}
\label{sec:orgebe8c06}

\begin{enumerate}
\item État des résultats
\label{sec:org5c7fff5}
\begin{enumerate}
\item Tableau
\label{sec:orga48a191}
\begin{center}
\begin{tabular}{ll}
<Nom compagnie> & \\
État des résultats & \\
du j j-mm-aaaa au jj-mm-aaaa & \\
\hline
\uline{produits} & \\
+ Ventes & x\\
- Couts des ventes & (x)\\
\textbf{Résultats bruts} & XX\\
+ Produits de location & x\\
+ Produits d'intérêts & x\\
+ Gains sur disposition (prix de vente - valeur comptable) & x\\
\uline{Charges} & \\
+ Écliectricité & x\\
+ Salaires & x\\
+ Assurances & x\\
+ Amortissement (Sans effet sur la trésorie) & x\\
+ Dépréciation des comptes client (Sans effet sur la trésorie) & x\\
+ Pertes sur disposition d'actif & x\\
\uline{Resultat net} & \\
= Sum(Produits) - Somme(Charges) & XX\\
\uline{Resultat net après impôts} & \\
= Resultat net - impot & XX\\
\end{tabular}
\end{center}
\item Vérifications
\label{sec:orge59aba6}
Les éléments ici influencent l'état de variations des capitaux propres et la
situation financière qui est vérifié avec l'équation comptable.
\end{enumerate}
\item État des variations des capitaux propres
\label{sec:org573ffca}
\begin{enumerate}
\item Tableau
\label{sec:orge1943d1}
\begin{center}
\begin{tabular}{llll}
<nom de la compagine> &  &  & \\
État des variations des capitaux propres &  &  & \\
Période terminée le <jj-mm-aa> &  &  & \\
\hline
 & Capital Social & RND & Surplus d'apport\\
\hline
Solde de début & a & b & c\\
\hline
Surplus d'apport de la période &  &  & h\\
Emission d'actions & d &  & \\
Rachat d'actions & (e) &  & \\
Resultats de la periode &  & f & \\
Dividendes déclarés &  & (g) & \\
\hline
Solde de fin & a + d - e & b + f - g & c + h\\
\hline
\end{tabular}
\end{center}

\item Explications
\label{sec:orgfa0cc85}
On prend a, b, c, de l'état des résultats.

Émettre des actions augmente le capital social.  Les RDN de fin est notre RND
précédent plus nos résultats de la période moins la partie des résultats qu'on a
versé en dividendes.  Finalement le surplus d'apport est le surplus d'apport n-1
plus le surplus d'apport que les propriétaires on mis durant la période.
\item Vérifications
\label{sec:org9b4c62c}
Les éléments ici influencent l'état de la situation financière qui est vérifié
avec l'équation comptable.
\end{enumerate}
\item État de la situation financière
\label{sec:org534d1be}
\begin{enumerate}
\item Tableau
\label{sec:org9939bbd}
\begin{center}
\begin{tabular}{lll}
\hline
<Nom de la compagnie> &  & \\
État de la situation financière &  & \\
Au 31 décembre 2017 &  & \\
\hline
 & 2017 & 2018\\
\hline
\uline{Actifs} &  & \\
\hline
\uline{Actifs courants} (< 12 mois) &  & \\
Encaisse & x & x\\
Stocks & x & x\\
Comptes Clients & x & x\\
(Provision pour dépréciation des CC) & (x) & (x)\\
Placements à court terme & x & x\\
Produits à recevoir & x & x\\
Charges payées d'avance & x & x\\
\textbf{Total actifs courants} & XX & XX\\
\hline
\uline{Actifs non-courants} &  & \\
Immobilisation corporelles &  & \\
Équipement/Machines & x & x\\
(Amortissement cumulé Équipement/Machine) & (x) & (x)\\
Immeubles & x & x\\
(Amortissement cumulé immeubles) & (x) & (x)\\
Immobilisations incorporelles (breuvets, licences) &  & \\
Placements à long terme & x & x\\
\textbf{Total actifs non-courants} & XX & XX\\
\hline
\textbf{Total Actifs} & XXX & XXX\\
\hline
Passifs et Capitaux propres &  & \\
\hline
\uline{Passifs courants} &  & \\
Comptes fournisseurs & x & x\\
Charges à payer & x & x\\
Dividendes à payer & x & x\\
Emprunts à court terme & x & x\\
Obligations à court terme & x & x\\
Produits différés (we owe merch to someone) & x & x\\
\textbf{Total pasifs courants} & XX & XX\\
\hline
\uline{Passifs non courants} &  & \\
Emprunts à long terme & x & x\\
Hypothèque & x & x\\
\textbf{Total passifs non courants} & XX & XX\\
\hline
\uline{Capitaux propres} &  & \\
Capital social & x & x\\
Résultats non distribués & x & x\\
Surplus d'apport & x & x\\
\textbf{Total Capitaux Propres} & XX & XX\\
\hline
\textbf{Total passifs et capitaux propres} & XXX & XXX\\
\hline
\end{tabular}
\end{center}
\item Composantes
\label{sec:org91264f8}
\begin{itemize}
\item Actifs(A):ressources économiques que l’entreprise possède ou sur lesquelles
elle exerce un contrôle et qui devraient lui procurer des avantages
économiques maintenant ou dans le futur.

\item Passifs(P):obligations qu’a l’entreprise envers des tiers et dont le
règlement se fera par transfert d’actifs, la prestation de services ou tout
autre avantage.

\item Capitauxpropres(C): mise(s) defonds du(des) propriétaire(s) auxquelles
s’ajoutent les résultats non distribués aux actionnaires sous forme de
dividendes (société par actions) et le surplus d’apport.
\end{itemize}
\item Vérifications:
\label{sec:orgdee228e}
Voir équation comptable A = P + CP
Les éléments ici influencent l'état des flux de trésorie qui a ses propres vérifications.
\item Remarques :
\label{sec:orga8bcced}
Document de synthèse qui expose à une date donnée la situation financière d'une
entreprise en fournissant un résumé de l'ensemble de ses éléments d'actif (A),
de passif(P)et de capitaux propres (C).

\begin{enumerate}
\item La situation financière se compose de deux parties principales: les
éléments d'actif et les sources d'actif (passif et capitaux propres).
\item Le total des éléments d'actif doit toujours égaler le total des sources
d'actif.
L'ÉQUATION COMPTABLE : A = P + C doit toujours être vérifiée.
\item À tout moment, il est possible d'établir les capitaux propres d'une
entreprise.
A - P = C
\end{enumerate}
\item Équation comptable
\label{sec:org4a9df22}
L'ÉQUATION COMPTABLE : A = P + C doit toujours être vérifiée.
\end{enumerate}

\item État des flux de trésorie (prochain cours)
\label{sec:org34edff1}
\begin{enumerate}
\item Tableau
\label{sec:org8e1fa2f}
\begin{center}
\begin{tabular}{ll}
\hline
Compagnie YY & \\
État des flux de trésorie (méthode indirecte) & \\
Du 1er Janvier au 31 décembre 2018 & \\
\hline
Activités de opérationnelles & \\
\hline
+ Resultat net & x\\
+ Éléments sans effets sur la trésorie & x\\
(amortissement) & \\
(Dépréciation des comptes clients) & \\
+ Delta PC (Fin - Début) (exclure prov pour dépréc CC) & x\\
+ Delta AC (Début - Fin) (exclure encaisse et placements CT) & x\\
\hline
\textbf{Total Activités Opérationnelles} & XX\\
\hline
Activité d'investissement & \\
\hline
- Achat d'actifs non courant & (x)\\
+ Vente d'actif non courant & x\\
+ Placements LT: (Debut - fin - perte + gains) & x\\
\hline
\textbf{Total Activités d'Investissement} & XX\\
\hline
Activités de financement & \\
\hline
+ Émission d'actions & x\\
- Rachat d'actions & (x)\\
- Versement de dividendes & (x)\\
+ Nouvel emprunts & x\\
- Remboursement d'emprunts & (x)\\
- Frais financiers & (x)\\
\hline
\textbf{Total activités de financement} & XX\\
\hline
Variation de la trésorie & XXX\\
Trésorie de début & XX\\
Trésorie de fin & XX\\
\hline
\end{tabular}
\end{center}

\item Vérifications
\label{sec:org3a9c411}
\begin{quote}
\(\Delta_{\text{Tr}}\) = \(\Sigma_{\text{AO}}\) + \(\Sigma_{\text{AF}}\) + \(\Sigma_{\text{AI}}\)
= \(\Delta_{\text{Encaisse}}\) + \(\Delta_{\text{PlacementsCT}}\)

Tr\(_{\text{Début}}\) = Encaisse\(_{\text{Début}}\) + PlacementsCT\(_{\text{Début}}\)

Tr\(_{\text{Fin}}\) = Encaisse\(_{\text{Fin}}\) + PlacementsCT\(_{\text{Fin}}\)
\end{quote}
\end{enumerate}
\end{enumerate}

\subsubsection{Ratios : Voir \href{https://moodle.polymtl.ca/pluginfile.php/512706/mod\_folder/content/0/Cours\%25202.1\%2520-\%2520Ratios.pdf?forcedownload=0}{Cours 2.1 Ratios.pdf}}
\label{sec:orgf1f3431}
\begin{enumerate}
\item Ratios de liquidité
\label{sec:org395282c}
\begin{enumerate}
\item Ratio de liquidité courante (ratio du fond de roulement)
\label{sec:org009079d}
\begin{verbatim}
actif courants
---------------
passif courants
\end{verbatim}
\begin{enumerate}
\item Interprétation
\label{sec:orgdf7b951}
\textbf{Capacité d'une entreprise à faire face à ses obligations financières à court
terme lorsqu'elles viennent à échéance}
\begin{itemize}
\item < 1: la compagnie a de la misère à rembourser ses dettes
\item > 2: L'entreprise a trop de stocks
\item\relax [1.2, 2.0] : La majorité des analystes estiment que c'est idéal
\end{itemize}
\end{enumerate}
\item Ratio de liquidité relative (Ratio de liquidité immédiate)
\label{sec:orgd733b77}
\begin{verbatim}
actif courant - stocks - frais payés d'avance
---------------------------------------------
              passif courant
\end{verbatim}
\begin{enumerate}
\item Interprétation
\label{sec:org72e710f}
*Indique si on peut rembourser le passif à court terme sans avoir à vendre les
stocks. Si le ratio est nettement inférieur au ratio de liquidé courante, cela
signifie que l'actif à court terme dépend étroitement des stocks.*
\begin{itemize}
\item Idéal : [1,2]
\end{itemize}
\end{enumerate}
\end{enumerate}
\item Ratios d'endettement
\label{sec:orgee7a5ee}
\begin{enumerate}
\item Ratio d'endettement
\label{sec:org3080583}
\begin{verbatim}
passif total
------------
actif total
\end{verbatim}
\begin{enumerate}
\item Interprétation
\label{sec:orgf5abbad}
Capacité de respecter ses engagements à long terme
\begin{itemize}
\item < 30\% : Excellent
\item\relax [30\%, 36\%] : bon
\item > 40\% : Problématique
\end{itemize}
\end{enumerate}
\item Ratio de la couverture des intérêts
\label{sec:orge3af0d2}
\begin{verbatim}
Résultat avant intérêts et impôts
---------------------------------
       Intérêts
\end{verbatim}
\begin{enumerate}
\item Interprétation
\label{sec:org9e6b93c}
Indique dans quelle mesure les intérêts débiteurs sont couverts par les flux de
trésorie de la société.
\begin{itemize}
\item < 1 : La société peut éprouver de véritables difficultés à régler ses intérêts
débiteurs et le risque de défaut de paiement est jugé élevé.
\item > 1.5 : Idéal
\end{itemize}
\end{enumerate}
\end{enumerate}
\item Ratios de performance
\label{sec:org4b0a0ac}
\begin{enumerate}
\item Ratio de rotation de l'actif
\label{sec:org277d6ca}
\begin{verbatim}
      Ventes nettes
-------------------------
Valeur moyenne de l'actif

Valeur moyenne = (montant de fin + montant de début) / 2
Ventes nettes = ventes moins les rendus, rabais et escomptes sur les ventes
\end{verbatim}
\begin{enumerate}
\item Interprétation
\label{sec:org7f79466}
\begin{itemize}
\item Il indique le montant du chiffre d'affaires généré par chaque dollar investi
dans l'actif total.
\item Habituellement, plus le ratio est élevé, plus la gestion des actifs est
optimale.
\item Par exemple, un ratio de 0,60 \$ indique que chaque dollar investi rapporte
0,60 \$ de ventes nettes.
\item Ce ratio est utile pour se comparer à ses concurrents.
\end{itemize}
\end{enumerate}
\item Ratio de rotation des comptes clients
\label{sec:org939bf8e}
\begin{verbatim}
         ventes nettes
----------------------------------
Valeur moyenne des comptes clients

Valeur moyenne = (montant de fin + montant de début) / 2
Ventes nettes = ventes moins les rendus, rabais et escomptes sur les ventes
\end{verbatim}
\begin{enumerate}
\item Interprétation
\label{sec:org75eb312}
\begin{itemize}
\item Il mesure le nombre de fois qu’une entreprise transforme ses créances clients en
ventes.
\item Plus ce ratio est faible, plus l’entreprise s’expose aux risques de non-paiement de la
part de ses partenaires, et plus elle devra revoir sa politique de crédit. Il s’agit donc,
pour l’entreprise, de trouver un juste milieu entre les créances accordées et les ventes
à générer via le crédit.
\item Un ratio de 15 signifie que l’entreprise collecte 15 fois ses compte-client par année. On
peut également conclure que ça lui prend en moyenne 24,3 jours pour collecter ses
clients (ratio du délai de recouvrement des c.c. 365/15).
\end{itemize}
\end{enumerate}
\item Ratio du délai de recouvrement des comptes clients
\label{sec:orgd19c756}
\begin{verbatim}
         365 jours
----------------------------
Rotation des Comptes Clients
\end{verbatim}
\begin{enumerate}
\item Interprétation
\label{sec:orgf44fcbc}
\begin{itemize}
\item Mesure le nombre moyen de jours qu’il faut aux clients pour payer leurs comptes.
\item Donne une indication de l’efficacité des politiques de crédit et de recouvrement en
vigueur dans l’entreprise.
\item Permet de déterminer si les modalités de crédit dont se sert l’entreprise sont réalistes.
\item Pour déterminer si le délai moyen de recouvrement des comptes clients est adéquat, il
suffit de le comparer aux modalités de crédit que vous offrez à vos clients.
\begin{itemize}
\item Par exemple, si vous donnez à vos clients 30 jours pour payer leur facture et que
votre délai moyen de recouvrement des comptes clients est de 45 jours, il y a un
problème; en revanche, si le délai moyen est inférieur à 30 jours, c’est une bonne
nouvelle.
\end{itemize}
\end{itemize}
\end{enumerate}
\item Ratio de rotation stocks
\label{sec:org00c3386}
\begin{verbatim}
    Coûts des ventes
-------------------------
Valeur moyenne des stocks
\end{verbatim}
\begin{enumerate}
\item Interprétation
\label{sec:org1ab4672}
\begin{itemize}
\item Un faible ratio de rotation est généralement de mauvais augure, car les produits ont
tendance à s'abîmer s'ils restent trop longtemps dans un entrepôt.
\item Les sociétés qui vendent des biens périssables ont un ratio de rotation très élevé.
\item Ce ratio est utile pour se comparer à ses concurrents.
\item Voici un lien utile afin de pouvoir comparer le ratio de votre entreprise à celui de votre
secteur d’activité :
\url{https://www.bdc.ca/fr/articles-outils/boite-outils-entrepreneur/evaluation-entreprise/pages/rotation-stocks-outil-analyse-comparative.aspx?ChangeIndustry=1}
\end{itemize}
\end{enumerate}
\item Ratio de rotation de l'actif immobilisé
\label{sec:org9866e1e}
\begin{verbatim}
            ventes nettes
--------------------------------------
valeur moyenne des actifs non courants

Valeur moyenne = (montant de fin + montant de début) / 2
Ventes nettes = ventes moins les rendus, rabais et escomptes sur les ventes
\end{verbatim}
\begin{enumerate}
\item Interprétation
\label{sec:orgc66b44b}
\begin{itemize}
\item Le ratio de rotation des immobilisations (rotation de l’actif immobilisé)
indique combien de revenus génère chaque dollar investi dans les
immobilisations.
\item Habituellement, plus le ratio est élevé, plus la gestion des actifs
immobilisés est optimale.
\item Attention à l’interprétation du ratio: Une entreprise bien installée possédant
de vieux équipements fortement amortie aurait un meilleur résultat en
opposition à une plus jeune entreprise possédant de nouveaux équipements.
\item Il faut donc comparer des entreprises possédant des immobilisations d’âge
comparable.
\end{itemize}
\end{enumerate}
\end{enumerate}
\item Ratios de rentabilité (rendement)
\label{sec:org40fa219}
\begin{enumerate}
\item Ratio de la marge nette
\label{sec:org43fa970}
\begin{verbatim}
Résultats nets
--------------
Ventes nettes

Ventes nettes : ventes moins les rendus, rabais et escomptes sur les ventes
Résultat net : résultat brut moins les charges
Résultat brut : ventes moins le coût des vente
\end{verbatim}
\begin{enumerate}
\item Interprétation
\label{sec:org0535613}
Rentabilité de l'exploitation
Ce ratio indique la partie des ventes qui contribue au bénéfice de l'entreprise.
\begin{itemize}
\item Le ratio ne sert à rien si l'entreprise perd de l'argent, car elle ne fait
alors pas de profit.
\item Lorsque le ratio de marge bénéficiaire nette (marge nette) est faible, cela
peut être attribuable à la stratégie prix et/ou à l'incidence que la
concurrence a sur la marge.
\item Une marge élevée est un bon signe
\item Ce ratio est utile pour se comparer à ses concurrents.
\end{itemize}
\end{enumerate}
\item Ratio de la marge brute
\label{sec:orgb9ddc11}
\begin{verbatim}
Resultats bruts
---------------
 Ventes nettes
\end{verbatim}
\begin{enumerate}
\item Interprétation
\label{sec:org6e38c28}
Indique le niveau de bénéfice généré par l'entreprise.

Ce ratio témoigne de la politique de fixation des prix de la société et de la majoration
réelle des prix.

Par exemple, si le ration de la marge bénéficiaire brute (marge brute) est de 33 \%,
cela signifie que le prix des produits est majoré de 50 \% (1/(1-33 \%))-100 \%).
\begin{itemize}
\item Les résultats peuvent être biaisés si l'éventail de produits de la société est
très large.
\item Ce ratio se révèle très utile lorsqu'il est comparé à ceux des exercices
antérieurs.
\item En général, la marge bénéficiaire brute doit être stable. Elle ne doit pas
fluctuer d'une période à l'autre.
\end{itemize}
\end{enumerate}
\item Ratio de rentabilité
\label{sec:orgb1f96b0}
\begin{verbatim}
         résultats nets
-----------------------------------
valeur moyenne des capitaux propres

Valeur moyenne = (Montant de fin + montant début)/2
Résultat net = résultat brut moins les charges
\end{verbatim}
\begin{enumerate}
\item Interprétation
\label{sec:org4744751}
Ce ratio indique le taux de rendement que l'entreprise tire de l'investissement
de ses propriétaires ou que les actionnaires obtiennent de leurs actions.

Par exemple, si le ratio s'élève à 10 \%, cela signifie que chaque dollar investi
à l'origine produit un actif de 10 cents.

\begin{itemize}
\item Les sociétés à forte croissance devraient obtenir un rendement élevé des capitaux
propres.
\item Le rendement moyen des capitaux propres au cours des 5 à 10 dernières années
donne une meilleure idée de la croissance à long terme.
\end{itemize}
\end{enumerate}
\end{enumerate}

\item Ratios de la valeur marchande
\label{sec:orgae5062a}
\begin{enumerate}
\item Ratio du résultat par action
\label{sec:orgcec9e1c}
\begin{verbatim}
            Résultats nets
------------------------------------------
Nombre d'actions ordinaires en circulation

Résultats net = résultat brut moins les charges
\end{verbatim}
\begin{enumerate}
\item Interprétation
\label{sec:org3292954}
C'est le ratio le plus utilisé. Il indique le montant du bénéfice généré par la
société, par action.
\begin{itemize}
\item On compare ce ratio avec les exercices précédents ainsi qu’avec d’autres
entreprises.
\item Lorsque la société émet de nouvelles actions, il est beaucoup plus difficile
de comparer le BPA de l'exercice en cours et des exercices précédents.
\item Le ratio du bénéfice par action (BPA ou du résultat par action) est
principalement utilisé pour les sociétés cotées en bourse. En soi, le BPA
n'indique pas grand-chose. Par contre, si vous le comparez au BPA d'un
trimestre ou d'un exercice antérieur, vous pouvez calculer le taux de
croissance du bénéfice (par action) de la société. Par exemple, une entreprise
qui augmente de 50 \% en un an possède un excellent taux de croissance.
\end{itemize}
\end{enumerate}
\item Ratio du cours/résultats
\label{sec:org8fe9b24}
\begin{verbatim}
Cours du marché ordinaire
-------------------------
Résultat par action

Résultats net = résultat brut moins les charges
\end{verbatim}
\begin{enumerate}
\item Interprétation
\label{sec:orgcea1f09}
L'un des ratios les plus utilisés, il permet de comparer le prix courant au
bénéfice afin de déterminer si le titre est surévalué ou sous-évalué. Il
représente une attente quant au rendement futur de l'entreprise.
\begin{itemize}
\item En règle générale, lorsque le ratio est élevé, cela signifie que les
investisseurs prévoient une forte croissance dans le futur.
\item Le ratio cours/bénéfice (cours/résultats, C/B) moyen du marché est de 20 à 25
fois le bénéfice.
\item Les sociétés qui perdent de l'argent n'ont pas de ratio C/B.
\item On compare ce ratio avec les exercices précédents ainsi qu’avec d’autres
entreprises.
\item Le ratio C/B ne détermine pas le cours de l'action. Un ratio C/B bas peut
indiquer que le bénéfice de l'entreprise est stable ou qu'il croît lentement,
comme il peut signifier que l'entreprise éprouve des difficultés financières.
\end{itemize}
\end{enumerate}
\end{enumerate}
\end{enumerate}
\subsubsection{Immobilisations et amortissement}
\label{sec:org8cead7e}
\begin{enumerate}
\item Vocabulaire
\label{sec:orge60d776}
\textbf{Actif immobilisé/Immobilisations} : 
\begin{itemize}
\item Destinés à être utilisés pour la production de bien ou pour gagner du revenu.
\item Destinés à être utilisés de façon durable
\item Non destinés à être vendus
\end{itemize}
\begin{enumerate}
\item Catégories d'immobilisations
\label{sec:orgbca96a3}
\begin{enumerate}
\item Immobilisations corporelles
\label{sec:org59a97ae}
\begin{itemize}
\item Biens amortissables : 
\begin{itemize}
\item Immeubles
\item Biens qui s'épuisent
\end{itemize}
\item Biens non-ammortissables : 
\begin{itemize}
\item Terrain
\end{itemize}
\end{itemize}
\item Immobilisations incorporelles
\label{sec:org71d7741}
\begin{itemize}
\item R\&D
\item Droits d'auteur
\item Franchises
\item Licenses
\item Marque de commerce
\item Autres propriétés intellectuelles
\end{itemize}
\end{enumerate}
\end{enumerate}

\item Coût d'un actif immobilisé
\label{sec:orgcdd165a}
L'idée est de prendre en compte 
\begin{center}
\begin{tabular}{l}
Coût de l'actif (prix affiché)\\
+ Coûts de mise en service\\
+ Autres frais comme transport, préparation, installation\\
\end{tabular}
\end{center}

\item Coût Amortissable
\label{sec:orgb3af773}

Si on achète un terrain et un immeuble pour un prix P, il faut déterminer quelle
partie est amortissable.

Une évaluation municipale par exemple dirait que l'immeuble vaut I et le terrain
vaut T.  Alors on va dire que P * (I/T) est le coût amortissable de l'immeuble.

Notons que nous n'avons pas nécéssairement I + T = P.  Nous utilisons les
évaluations pour obtenir un ratio de valeur, mais nous utilisons le coût d'achat
comme entrée pour avoir le coût amortissable.

\item Amortissement
\label{sec:org4eec26d}
\begin{enumerate}
\item Amortissement linéaire
\label{sec:orgc807cbb}
L'immobilisation passe du coût initial \textbf{P} à sa valeur résiduelle \textbf{R} en \textbf{n}
années en perdant \textbf{(P-R)/n} à chaque année (sauf la première année où on prend
une fraction de ce montant correspondant à la fraction d'année d'utilisation(m/12)).

D\(_{\text{1}}\) = (m/12) * (P-R)/n

D\(_{\text{t}}\) = (P-R)/n

\item Amortissement à taux dégressif à taux constant
\label{sec:orge8e164f}

Aller de \textbf{P} à \textbf{R} en \textbf{n} étapes en multipliant par un facteur.

L'idée est en fait de multiplier la valeur de l'année précédente par un facteur
de sorte qu'après avoir multiplié par ce facteur \textbf{n} fois, on fini par avoir
multiplié par \textbf{R/P}.

\begin{center}
\begin{tabular}{rl}
Année & Valeur comptable\\
\hline
0 & P\\
1 & P (R/P)\(^{\text{1/n}}\)\\
2 & P (R/P)\(^{\text{1/n}}\) (R/P)\(^{\text{1/n}}\)\\
\dots{} & \dots{}\\
n & P ( (R/P)\(^{\text{1/n}}\) )\(^{\text{n}}\) = P (R/P) = R\\
\end{tabular}
\end{center}


En pratique, on utilise la terminologie suivante:
On calcule un taux basé sur \textbf{P}, \textbf{R} et \textbf{n}:

d = 1 - (R/P)\(^{\text{1/n}}\)

et au lieu de soustraire le même montant à chaque année, on ammortit de

D\(_{\text{t}}\) = CNA\(_{\text{t-1}}\) * d

Ce qui correspond à multiplier par (R/P)\(^{\text{1/n}}\):

CNA\(_{\text{t}}\) = CNA\(_{\text{t-1}}\) - CNA\(_{\text{t-1}}\) * d = CNA\(_{\text{t-1}}\) * (1 - d) = CNA\(_{\text{t-1}}\) * (R/P)\(^{\text{1/n}}\)

\item Amortissement Proportionnel à l'ordre inversé des années
\label{sec:org4cc01ea}

Soit U\(_{\text{t}}\) = (P-R)(n - t + 1)/k la perte de valeur dans la première année \textbf{d'utilisation}.

Si les années d'utilisation sont désynchronisées avec les périodes d'éxercice,
chaque année d'exercice ira chercher une partie de chaque année d'utilisation qui
la chevauchent.

Par exemple, pour une machinne achetée en fin avril de l'année 1: on a
U\(_{\text{1}}\) = (P-R)(n-1+1)/k, mais seulement (8/12)U\(_{\text{1}}\) sera mis dans les livres pour
l'année 1.

D\(_{\text{1}}\) = (8/12)U\(_{\text{1}}\)

À l'année fiscale 2, on a 4 mois qui font partie de la première année
d'utilisation (U\(_{\text{1}}\)).  L'amortissement à l'année 2 incluera (4/12)D\(_{\text{1}}\) et aussi (8/12)D\(_{\text{2}}\)
où

D\(_{\text{2}}\) = (4/12)U\(_{\text{1}}\) + (8/12)U\(_{\text{2}}\)

\textbf{En passant, k = n(n+1)/2.}

/Sti que ça vaut pas la peine d'apprendre cet esti d'amortissement, come on.
R'garde, si j'ai besoin de faire un amortissement comme ça, j'ouvrirai un
livre./


\item Amortissement Proportionnel à l'utilisation
\label{sec:orgc9fe23a}

Ben simple: on remplace le \textbf{n} par un volume total d'utilisation \textbf{V}.

on a donc d = (P-R)/V

L'amortissement à l'année \textbf{t} sera donné par le volume d'utilisation de v\(_{\text{t}}\) de
l'année \textbf{t}.

D\(_{\text{t}}\) = d*v\(_{\text{t}}\)

Par exemple, ça pourrait être une voiture, \textbf{V} serait le nombre total de
kilomètres qu'on prévoit faire avec la voiture et v\(_{\text{t}}\) serait le nombre de
kilomètres faits durant l'année \textbf{t}.
\end{enumerate}
\end{enumerate}

\subsubsection{Intérêt et valeur de l'argent dans le temps}
\label{sec:org4cf2016}
À cause des intérêts ou de l'inflation, un montant X\(_{\text{1}}\) à un temps T\(_{\text{1}}\) n'aura pas
la même valeur à un temps T\(_{\text{2}}\).  Les méthodes dans cette sections permettent de
trouver des équivalences entre des montants d'argents à différents temps.

Les facteurs d'actualisations permettent d'établire ces équivalences.

Les taux effectifs permettent de comparer des situations avec des fréquences de
capitalisations différentes.
\begin{enumerate}
\item Valeur de l'argent dans le temps
\label{sec:org9560b6b}

\begin{enumerate}
\item La base
\label{sec:org323ed20}

Avec un taux i, un montant est multiplié par (1+i) à chaque période de
capitalisation.

Ainsi, un montant à l'année 0 aura été mutiplié par (1+i)\(^{\text{n}}\) rendu à l'année n.

Ainsi, si on connait le montant à l'année n, on peut connaître le montant à
l'année 0 en multipliant par (1+i)\(^{\text{-n}}\).  Donc:

\begin{center}
\begin{tabular}{ll}
(P/F, i, n) & (1+i)\(^{\text{-n}}\)\\
(F/P, i, n) & (1+i)\(^{\text{n}}\)\\
\end{tabular}
\end{center}

\item Le reste
\label{sec:orgc9a92e4}

Avec des annuités, on dépose un montant A à chaque période. Au bout de n années,
le premier montant A déposé à l'année vaudrait A(1+i)\(^{\text{-1}}\) à l'année 0, le montant
déposé à l'année 2 vaudrait A(1+i)\(^{\text{-2}}\) à l'année 0.  Finalement le montant déposé
à l'année n vaudrait A(1+i)\(^{\text{-n}}\) à l'année 0.

La somme est donc

A((1+i)\(^{\text{-1}}\) + (1+i)\(^{\text{-2}}\) + \dots{} + (1+i)\(^{\text{-n}}\)) = P

Si on fait le même truc qu'avec une série géométrique pour trouver P/A = \(\xi\), nous
avons

\(\xi\) = (1+i)\(^{\text{-1}}\) + (1+i)\(^{\text{-2}}\) + \dots{} + (1+i)\(^{\text{-n}}\)

et

(1+i)\(\xi\) = 1 + (1+i)\(^{\text{-1}}\) + (1+i)\(^{\text{-2}}\) + \dots{} + (1+i)\(^{\text{-(n-1)}}\) 

donc \(\xi\) - (1+i)\(\xi\) = -i\(\xi\) = (1+i)\(^{\text{-n}}\) - 1.

ou bien i\(\xi\) = 1 - (1+i)\(^{\text{-n}}\) et donc \(\xi\) = (1 - (1+i)\(^{\text{-n}}\))/i

On peut multiplier par (1+i)\(^{\text{n}}\)/(1+i)\(^{\text{n}}\) = 1 pour obtenir

\begin{center}
\begin{tabular}{ll}
(P/A, i, n) & ((1+i)\(^{\text{n}}\) - 1)/(i(1+i)\(^{\text{n}}\))\\
(A/P, i, n) & 1/(P/A, i, n)\\
\end{tabular}
\end{center}

pour avoir une formule sans exposants négatifs.

Les autres formules peuvent être obtenues ainsi:

\begin{center}
\begin{tabular}{ll}
(F/A, i, n) & ((1+i)\(^{\text{n}}\) - 1)/i\\
(A/F, i, n) & i / ((1+i)\(^{\text{n}}\) - 1)\\
(P/G, i, n) & \ldots{}\\
( & \\
\end{tabular}
\end{center}

\item Cas spécial: Annuité à l'infini
\label{sec:orgbbfa955}

Combien d'argent dois-je déposer dans un compte pour pouvoir retirer A\(_{\infty}\) à
chaque année jusqu'à la fin des temps?  Pour que ça marche, c'est que les
intérêts qui me seront versés à chaque année vont être l'annuité.

Si je place 1000 à un taux \textbf{i}, à chaque année, la banque va me donner
1000 \(\cdot\) i.  Si je retire 1000 \(\cdot\) i, il me reste encore 1000 et je pourrai retirer
1000 \(\cdot\) i l'année prochaine.

Donc si je veux retirer A\(_{\infty}\) par année jusqu'à la fin des temps, il faut que

A\(_{\infty}\) = i \(\cdot\) P.

\textbf{Tout ce que c'est les annuités à l'infini, c'est de dire que je vais me payer
par l'intérêt à chaque année.}

Ou en d'autres termes, j'ai un projet qui me rapporte 100\$ par année jusqu'à la
fin des temps.  Et le taux est de 7\%.  Ça veut dire que la valeur actuelle de ce
projet est de 100/0.07 = 1428.57 parce que le projet est équivalent à mettre
1428.57 dans un compte avec un rendement de 7\% et je vais recevoir 100\$ par
année comme ce que le projet me donnerait.

Mathématiquement, notons que

(P/A, i, n) = (1 - (1+i)\(^{\text{-n}}\))/i

et que lorsque n \(\to\) \(\infty\), ceci tend vers (1+0)/i = 1/i. Notons que ceci requiert que
i > 0 pour que (1+i) > 1 pour que (1+i)\(^{\text{-n}}\) \(\to\) 0 quand n \(\to\) \(\infty\).
\end{enumerate}

\item Taux nominal et périodes de capitalisation
\label{sec:org203e86f}

Les termes d'un prêt ou d'un investissement expriment les taux sous la forme

"Un taux nominal de i\(_{\text{nominal}}\) par année capitalisée m fois par année"

Ceci veux dire que m fois par année, le montant sera multiplié par (1 + i/m).
Au bout d'une année, le montant aura été multiplié par (1 + i/m)\(^{\text{m}}\).

On appelle i\(_{\text{eff}}\) = (1 + i\(_{\text{nom}}\)/m)\(^{\text{m}}\).

Afin de pouvoir travailler avec des flux monétaires avec diverses fréquences de
capitalisation, nous devons transformer ces taux en taux effectifs de la même
fréquence (généralement on met tout en années).

Périodes de versements:

On a vu une formule pour avoir un taux effectif qui a un \textbf{m} et un \textbf{v}.  Ce
qu'on veut, c'est que

(1+i/m)\(^{\text{m}}\) = (1 + i/v)\(^{\text{v}}\)

On fait X\(^{\text{(1/v)}}\) des deux côtés pour obtenir

((1+i\(_{\text{1}}\)/m)\(^{\text{m}}\))\(^{\text{1/v}}\) = 1 + i\(_{\text{2}}\)/v \\
\(\Rightarrow\) (1 + i\(_{\text{1}}\)/m)\(^{\text{m/v}}\) = 1 + i\(_{\text{2}}\)/v \\
\(\Rightarrow\) (1 + i\(_{\text{1}}\)/m)\(^{\text{m/v}}\) - 1 = i\(_{\text{2}}\)/v \\

Exemple: un taux de 10\% capitalisé mensuellement mais payés semestriellement
donne:

(1 + 0.10/12)\(^{\text{12}}\) = (1 + i\(_{\text{2}}\)/2)\(^{\text{2}}\)      \\
\(\Rightarrow\) (1 + 0.10/12)\(^{\text{6}}\) = 1 + i\(_{\text{2}}\)/2       \\
\(\Rightarrow\) (1 + 0.10/12)\(^{\text{6}}\) - 1 = 0.05105    \\

Ce qui veut dire que notre taux de 10\% par année capitalisé mensuellement
correspond à un taux de 5.105\% effectif par période de paiement.

Ça correspond aussi à un taux de 10.21\% capitalisé semestriellement.

En effet:

(1 + 0.10/12)\(^{\text{12}}\) = 1.1047

et 

(1 + 0.1021/2)\(^{\text{2}}\) = 1.1047

Le 0.05105, c'est le i\(_{\text{2}}\)/v. C'est le taux effectif.  Si on veut un taux nominal
annuel, c'est i\(_{\text{2}}\) = 10.21\%.
\end{enumerate}

\subsubsection{Interpolations:}
\label{sec:org13f1555}

Supposons une fonction y = f(x) et nous cherchons x\(^{\text{*}}\) pour que f(x\(^{\text{*}}\)) = y\(^{\text{*}}\),

Si on a x\(_{\text{1}}\), x\(_{\text{2}}\) tels que f(x\(_{\text{1}}\)) < y\(^{\text{*}}\) < f(x\(_{\text{2}}\)) (sans dire si x\(_{\text{1}}\) > x\(_{\text{2}}\) ou x\(_{\text{2}}\) > x\(_{\text{2}}\)), la
droite reliant (x\(_{\text{2}}\), y\(_{\text{2}}\)) à (x\(^{\text{*}}\), y\(^{\text{*}}\)) a la même pente que la droite reliant (x\(_{\text{1}}\),
y\(_{\text{1}}\)) à (x\(_{\text{2}}\), y\(_{\text{2}}\)) donc

\begin{verbatim}
x_2 - x^*     x_2 - x_1  \\
---------  =  ---------  \\
y_2 - y^*     y_2 - y_1  \\

Et on peut manipuler cette équation

x_1 - x^*    x_2 - x_1            \\
---------  = --------- (y_1 - y^*)  \\
             y_2 - y_1            \\


              x_2 - x_1            \\
-x^* = -x_1 + --------- (y_1 - y^*)  \\
              y_2 - y_1            \\

pour obtenir

              x_2 - x_1            \\
 x^* = +x_1 - --------- (y_1 - y^*)  \\
              y_2 - y_1            \\

Et notons que dans les slides, c'est généralement présenté comme suit:

              x_2 - x_1            \\
 x^* = +x_1 + --------- (y^* - y_1)  \\
              y_2 - y_1            \\
\end{verbatim}

x\(^{\text{*}}\) = +x\(_{\text{1}}\) + (y\(^{\text{*}}\) - y\(_{\text{1}}\)) \(\cdot\) (x\(_{\text{2}}\) - x\(_{\text{1}}\))/(y\(_{\text{2}}\) - y\(_{\text{1}}\)) \\

\subsection{Travaux Pratiques (TP)}
\label{sec:org4a8034a}


\subsection{Feuille de formules}
\label{sec:org32445f2}

\begin{itemize}
\item[{$\boxtimes$}] États financiers [4/4]
\begin{itemize}
\item[{$\boxtimes$}] État des résultats
\begin{itemize}
\item[{$\boxtimes$}] Notes
\end{itemize}
\item[{$\boxtimes$}] État de variations de CP
\begin{itemize}
\item[{$\boxtimes$}] Notes
\end{itemize}
\item[{$\boxtimes$}] État des résultats
\begin{itemize}
\item[{$\boxtimes$}] Notes
\end{itemize}
\item[{$\boxtimes$}] État des flux de trésorie
\begin{itemize}
\item[{$\boxtimes$}] Notes
\end{itemize}
\end{itemize}

\item[{$\boxminus$}] Ratios [4/14]

\begin{itemize}
\item[{$\boxtimes$}] 

\item[{$\boxtimes$}] 

\item[{$\square$}] 

\item[{$\square$}] 

\item[{$\boxtimes$}] 

\item[{$\boxtimes$}] 

\item[{$\square$}] 

\item[{$\square$}] 

\item[{$\square$}] 

\item[{$\square$}] 

\item[{$\square$}] 

\item[{$\square$}] 

\item[{$\square$}] 

\item[{$\square$}] 
\end{itemize}

\item[{$\square$}] Cas de figure (valeur de l'argent dans le temps) [0/5]
\begin{itemize}
\item[{$\square$}] Annuités
\item[{$\square$}] Gradient arithmétique
\item[{$\square$}] Gradient géométrique
\item[{$\square$}] Annuité à l'infini
\item[{$\square$}] Formule d'interpolation (Slides extra MVH P35)
\end{itemize}

\item[{$\square$}] Conversions de taux d'intérêts

\item[{$\square$}] Amortissement
\begin{itemize}
\item[{$\square$}] Note: Coût amortissable
\item[{$\square$}] Amortissement Linéaire
\item[{$\square$}] Amortissement Dégressif
\item[{$\square$}] Amortissement Proportionnel à l'ordre inversé des années
\item[{$\square$}] Amortissement Linéaire
\end{itemize}
\end{itemize}
\end{document}
